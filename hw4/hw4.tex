%        File: hw4.tex
%     Created: Thu Apr 28 10:00 PM 2016 C
% Last Change: Thu Apr 28 10:00 PM 2016 C
%

\documentclass[a4paper]{article}

\title{Math 8584 Homework 4}
\date{5/6/16}
\author{Trevor Steil}

\usepackage{amsmath}
\usepackage{amsthm}
\usepackage{amssymb}
\usepackage{esint}

\newtheorem{theorem}{Theorem}[section]
\newtheorem{corollary}{Corollary}[section]
\newtheorem{proposition}{Proposition}[section]
\newtheorem{lemma}{Lemma}[section]
\newtheorem*{claim}{Claim}
\newtheorem*{problem}{Problem}
%\newtheorem*{lemma}{Lemma}
\newtheorem{definition}{Definition}[section]

\newcommand{\R}{\mathbb{R}}
\newcommand{\N}{\mathbb{N}}
\newcommand{\C}{\mathbb{C}}
\newcommand{\Z}{\mathbb{Z}}
\newcommand{\supp}[1]{\mathop{\mathrm{supp}}\left(#1\right)}
\newcommand{\lip}[1]{\mathop{\mathrm{Lip}}\left(#1\right)}
\newcommand{\curl}{\mathrm{curl}}
\newcommand{\la}{\left \langle}
\newcommand{\ra}{\right \rangle}
\renewcommand{\vec}[1]{\mathbf{#1}}

\newenvironment{solution}{\emph{Solution.}}

\begin{document}
\maketitle
\begin{enumerate}
  \item McOwen 7.1.1
    \begin{problem}
      Let $\Omega$ be a bounded domain, $X = L^2(\Omega), Y = H_0^{1,2}(\Omega,$ and $G(u) = \int_{}^{}u^2 dx$.
      \begin{enumerate}
        \item Show that $G:X \to \R$ and $G:Y \to \R$ are countinuous functionals.

        \item Compute $G'(u)$.

        \item Show that $G$ is differentiable on both $X$ and $Y$.

        \item Show that $G$ is $C^1$ on both $X$ and $Y$.

      \end{enumerate}

    \end{problem}

    \begin{solution}

      \begin{enumerate}
        \item

        \item

        \item

        \item

      \end{enumerate}

    \end{solution}

  \item McOwen 7.1.3
    \begin{problem}
      \begin{enumerate}
        \item If $X$ is a Hilbert space and $u_j \rightharpoonup u$ weakly in $X$, then $\|u\|_X \leq \liminf_{j \to \infty} \|u_j\|X$.

        \item Show that weak upper semicontinuity for the norm on a Hilbert space need not be true: $u_j \rightharpoonup u \not \Rightarrow \|u\|_X
          \geq \limsup \|u_j\|_X$.

      \end{enumerate}

    \end{problem}

    \begin{solution}

      \begin{enumerate}
        \item

        \item

      \end{enumerate}

    \end{solution}

  \item McOwen 7.1.4

    \begin{problem}
      \begin{enumerate}
        \item If $a,b \in \R^n$, verify that $|a+b|^2 + |a|^2 \geq |b|^2/2$.

        \item Complete the argument for the existence of a minimizing function $u \in \mathcal{A}$ for $F$ in Example 2.
      \end{enumerate}

    \end{problem}

    \begin{solution}

      \begin{enumerate}
        \item

        \item

      \end{enumerate}

    \end{solution}

  \item McOwen 7.1.5
    \begin{problem}
      Suppose we want to solve $\Delta u = 0$ in $\Omega$ with $\partial u / \partial \nu = h$ on $\partial \Omega$, where $\Omega$ is a smooth
      bounded domain in $\R^n$ and $h \in C^\infty(\partial \Omega)$ satisfies $\int_{\partial \Omega}^{} h dS = 0$. Show how to obtain a weak
      solution $u \in H^{1,2}(\Omega)$ from the existence theory discussed in this section.
    \end{problem}

    \begin{solution}
    \end{solution}

  \item McOwen 7.1.6
    \begin{problem}
      Suppose $\Omega$ is a bounded domain, $q(x)$ is a bounded function on $\Omega$ satisfying $q(x) \leq \eta$, and $f \in L^2(\Omega)$. If $\eta
      \geq 0$ is sufficiently small, then show that the Dirichlet problem $\Delta u + q(x) u = f$ in $\Omega, u = 0$ on $\partial \Omega$ admits a
      weak solution.
    \end{problem}

    \begin{solution}
    \end{solution}

  \item McOwen 7.1.9
    \begin{problem}
      Let $\Omega$ be a smooth bounded domain in $\R^n, g \in H^{2,2}(\Omega)$, and $h \in H^{1,2}(\Omega)$. Show that a critical point of the
      functional
      \[ F(u) = \frac{1}{2} \int_{\Omega}^{} (\Delta u)^2 dx dy \]
      for $u \in \mathcal{A} \equiv \{ u \in H^{2,2}(\Omega) : u - g \in H^{2,2}_0(\Omega) \text{ and } \partial u / \partial \nu - h \in H_0^{1,2}
      (\Omega) \}$ is a weak solution of the biharmonic equation $\Delta^2 u = \Delta (\Delta u) = 0$ in $\Omega$, with Dirichlet boundary conditions
      $u = g$ and $\partial u / \partial \nu = h$ on $\partial \Omega$.
    \end{problem}

    \begin{solution}
    \end{solution}

  \item McOwen 7.2.2
    \begin{problem}
      In Theorem 2 of this section, we have not actually verified that every eigenvalue of
      \[ \begin{cases}
          \Delta u + \lambda u = 0 &\text{in } \Omega \\
          u = 0 &\text{on } \partial \Omega
      \end{cases} \]
      is one of the $\lambda_n$ generated by the functional $F$. Use the completeness of the eigenfuctions to prove this fact.
    \end{problem}

    \begin{solution}
    \end{solution}

  \item McOwen 7.2.3
    \begin{problem}
      Let $\lambda_1, \lambda_2, \dots$ be eigenvalues of the Laplacian with corresponding eigenfunctions $u_1, u_2, \dots$. Let $f \in
      H^{1,2}_0(\Omega)$ and $ \alpha_n = \la f, u_n \ra$. Then
      \[ \| \nabla f \|_2^2 = \sum_{n = 1}^\infty \alpha_n^2 \lambda_n .\]
    \end{problem}

    \begin{solution}
    \end{solution}

  \item McOwen 7.2.4
    \begin{problem}
      Consider a uniformly elliptic operator $L$ in divergence form, and let us pose the eigenvalue problem
      \[ \begin{cases}
          Lu + \lambda u = 0 &\text{in } \Omega \\
          u = 0 &\text{on } \partial \Omega
      \end{cases} \]

      \begin{enumerate}
        \item Give a variational formulation for the first eigenvalue $\lambda_1$. Observe that if $c(x) \leq c_0$, then $\lambda_1 > - c_0$.

        \item Show that the eigenvalues form a nondecreasing sequence $\lambda_1 \leq \lambda_2 \leq \dots$ tending to infinity.

        \item Show that each eigenvalue has a finite-dimensional eigenspace, and the collection of all eigenfunctions $u_n$ forms an orthonormal basis
          for $L^2( \Omega)$.
      \end{enumerate}

    \end{problem}

    \begin{solution}

      \begin{enumerate}
        \item

        \item

        \item
      \end{enumerate}

    \end{solution}

  \item McOwen Theorem 3, page 236
    \begin{problem}
      If $\widehat{\Omega} \subset \Omega$ are bounded domains, then $\lambda_n( \widehat{\Omega}) \geq \lambda_n (\Omega)$.
    \end{problem}

    \begin{solution}
    \end{solution}

\end{enumerate}<++>
\end{document}



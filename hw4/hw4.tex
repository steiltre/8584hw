%        File: hw4.tex
%     Created: Thu Apr 28 10:00 PM 2016 C
% Last Change: Thu Apr 28 10:00 PM 2016 C
%

\documentclass[a4paper]{article}

\title{Math 8584 Homework 4}
\date{5/6/16}
\author{Trevor Steil}

\usepackage{amsmath}
\usepackage{amsthm}
\usepackage{amssymb}
\usepackage{esint}

\newtheorem{theorem}{Theorem}[section]
\newtheorem{corollary}{Corollary}[section]
\newtheorem{proposition}{Proposition}[section]
\newtheorem{lemma}{Lemma}[section]
\newtheorem*{claim}{Claim}
\newtheorem*{problem}{Problem}
%\newtheorem*{lemma}{Lemma}
\newtheorem{definition}{Definition}[section]

\newcommand{\R}{\mathbb{R}}
\newcommand{\N}{\mathbb{N}}
\newcommand{\C}{\mathbb{C}}
\newcommand{\Z}{\mathbb{Z}}
\newcommand{\supp}[1]{\mathop{\mathrm{supp}}\left(#1\right)}
\newcommand{\lip}[1]{\mathop{\mathrm{Lip}}\left(#1\right)}
\newcommand{\curl}{\mathrm{curl}}
\newcommand{\la}{\left \langle}
\newcommand{\ra}{\right \rangle}
\renewcommand{\vec}[1]{\mathbf{#1}}

\newenvironment{solution}{\emph{Solution.}}

\begin{document}
\maketitle
\begin{enumerate}
  \item McOwen 7.1.1
    \begin{problem}
      Let $\Omega$ be a bounded domain, $X = L^2(\Omega), Y = H_0^{1}(\Omega),$ and $G(u) = \int_{}^{}u^2 dx$.
      \begin{enumerate}
        \item Show that $G:X \to \R$ and $G:Y \to \R$ are countinuous functionals.

        \item Compute $G'(u)$.

        \item Show that $G$ is differentiable on both $X$ and $Y$.

        \item Show that $G$ is $C^1$ on both $X$ and $Y$.

      \end{enumerate}

    \end{problem}

    \begin{solution}

      \begin{enumerate}
        \item
          Take $u \in X$. Then
          \begin{align*}
            |G(u)| &= \left| \int_{\Omega}^{} u^2 dx \right| \\
            &\leq \int_{\Omega}^{} |u|^2 dx \\
            &= \|u\|_{L^2(\Omega)}
          \end{align*}
          Therefore, $G$ is continuous on $X$.

          Take $u \in Y$. Similar to before, we have
          \begin{align*}
            |G(u)| &\leq \|u\|_{L^2(\Omega)} \\
            &\leq \|u\|_{H^1_0(\Omega)}
          \end{align*}

        \item
          Take $u,v \in X$. We compute
          \begin{align*}
            G'(u)v &= \lim_{\varepsilon \to 0} \frac{G(u+\varepsilon v) - G(u)}{\varepsilon} \\
            &= \lim_{\varepsilon \to 0} \frac{\int_{\Omega}^{} (u + \varepsilon v)^2 dx - \int_{\Omega}^{} u^2 dx}{\varepsilon} \\
            &= \lim_{\varepsilon \to 0} \frac{1}{\varepsilon} \int_{\Omega}^{} ( 2 \varepsilon uv + \varepsilon^2 v^2 ) dx \\
            &= \int_{\Omega}^{} \lim_{\varepsilon \to 0} ( 2 uv + \varepsilon v^2 ) dx \quad \parbox{5cm}{using H\"{o}lder's inequality and the fact
            that $u,v \in L^2(\Omega)$} \\
            &= \int_{\Omega}^{} 2 uv dx
          \end{align*}

        \item
          Take $u,v \in L^2(\Omega)$. Then
          \begin{align*}
            |G(u+v) - G(u) - G'(u)v| &= \int_{\Omega}^{} \left( (u+v)^2 - u^2 - 2uv \right) dx \\
            &= \int_{\Omega}^{} v^2 dx \\
            &= \|v\|_{L^2(\Omega)}
          \end{align*}
          Therefore, $|G(u+v) - G(u) - G'(u)v| \to 0$ as $\|v\|_X \to 0$ and $G$ is differentiable on $X$.

          Now take $u,v \in H_0^1(\Omega)$. Then similar to before, we get
          \begin{align*}
            |G(u+v) - G(u) - G'(u)v| &= \|v\|_L^2(\Omega) \\
            &= \frac{1}{2} \|v\|_{L^2(\Omega)} + \frac{1}{2} \|v\|_{L^2(\Omega)} \\
            &\leq \frac{1}{2} \|v\|_{L^2(\Omega)} + C \| \nabla v \|_{L^2(\Omega)} \quad \parbox{5cm}{by Poincar\'{e}'s inequality} \\
            &\leq C' \|v\|_{H_0^1(\Omega)}
          \end{align*}

          Therefore, $|G(u+v) - G(u) - G'(u)v| \to 0$ as $\|v\|_Y \to 0$ and $G$ is differentiable on $Y$.

        \item
          Let $u,v,w \in X$. Then
          \begin{align*}
            |(G'(u) - G'(v))w| &= 2 \left| \int_{\Omega}^{} (u-v)w dx \right| \\
            &\leq 2 \|u-v\|_{L^2(\Omega)} \|w\|_{L^2(\Omega)}
          \end{align*}

          Therefore,
          \begin{align*}
            \| G'(u) - G'(v) \|_{X^\ast} &= \sup_{w \not = 0} \frac{|(G'(u)-G'(v))w|}{\|w\|_{L^2(\Omega)}} \\
            &\leq \|u-v\|_{L^2(\Omega)}
          \end{align*}
          So $G'$ is continuous on $X$.

          Now take $u,v,w \in Y$. Using Poincar\'{e} as before, we get
          \begin{align*}
            | (G'(u) - G'(v))w| &\leq 2 \|u-v\|_{L^2(\Omega)} \|w\|_{L^2(\Omega)} \\
            &\leq C \|u-v\|_{H_0^1(\Omega)} \|w\|_{H_0^1(\Omega)}
          \end{align*}

          Therefore,
          \[ \|G'(u) - G'(v)\|_{Y^\ast} \leq C\|u-v\|_{H_0^1(\Omega)} \]
          and $G'$ is continuous on $Y$.

      \end{enumerate}

    \end{solution}

    \pagebreak

  \item McOwen 7.1.3
    \begin{problem}
      \begin{enumerate}
        \item If $X$ is a Hilbert space and $u_j \rightharpoonup u$ weakly in $X$, then $\|u\|_X \leq \liminf_{j \to \infty} \|u_j\|_X$.

        \item Show that weak upper semicontinuity for the norm on a Hilbert space need not be true: $u_j \rightharpoonup u \not \Rightarrow \|u\|_X
          \geq \limsup \|u_j\|_X$.

      \end{enumerate}

    \end{problem}

    \begin{solution}

      \begin{enumerate}
        \item
          By weak convergence, for any $v \in X$, $\la u_j, v \ra \to \la u, v \ra$ as $j \to \infty$. In particular, if we choose $v = u$, then
          \begin{align*}
            \|u\|_X^2 &= | \la u, u \ra | \\
            &= \liminf_{j \to \infty} | \la u_j, u \ra | \quad \parbox{5cm}{by weak convergence} \\
            &\leq \liminf_{j \to \infty} \|u_j\|_X \|u\|_X
          \end{align*}

          Therefore, $\|u\|_X \leq \liminf_{j \to \infty} \|u_j\|_X$.

        \item
          Let $\{u_j\}_{j=1}^\infty$ be an orthonormal set in $X$. Then for any $v \in X$, we have
          \[ \sum_j |\la u_j, v\ra|^2 \leq \|v\|_X^2 .\]
          Because this sequence converges for any $v \in X$, we must have
          \[ \la u_j, v \ra \to 0 .\]
          Therefore
          \[ u_j \rightharpoonup 0 .\]

          Because the $u_j$ form an orthonormal sequence, $\|u_j\|=1$ for all $j$. So
          \[ \|0\|_X = 0 \not \geq 1 = \limsup_{j} \|u_j\|_X .\]

      \end{enumerate}

    \end{solution}

  \item McOwen 7.1.4

    \begin{problem}
      \begin{enumerate}
        \item If $a,b \in \R^n$, verify that $|a+b|^2 + |a|^2 \geq |b|^2/2$.

        \item Complete the argument for the existence of a minimizing function $u \in \mathcal{A}$ for $F$ in Example 2.
      \end{enumerate}

    \end{problem}

    \begin{solution}

      \begin{enumerate}
        \item
          Take $a,b \in \R^n$. Then
          \begin{align*}
            0 &\leq \left| \sqrt{2} a + \frac{1}{\sqrt{2}}b \right|^2 \\
            &= 2 |a|^2 + 2a \cdot b + \frac{1}{2} |b|^2 \\
            &= |a + b|^2 + |a|^2 - \frac{1}{2} |b|^2
          \end{align*}

          Therefore,
          \[ |a+b|^2 + |a|^2 \geq \frac{1}{2} |b|^2 .\]

        \item
          We have the functional
          \[ F(u) = \frac{1}{2} \int_{\Omega}^{} | \nabla u|^2 dx \]
          that we wish to minimize over the set
          \[ \mathcal{A} = \{ u \in H^1(\Omega) : u-g \in H^1_0(\Omega) \} = \{ u = g+v : v \in H^1_0(\Omega) \} .\]

          Take $u \in H^1_0(\Omega)$. Then
          \begin{align*}
            F(u) &= \frac{1}{2} \int_{\Omega}^{} | \nabla u |^2 dx \\
            &= \frac{1}{4} \int_{\Omega}^{} | \nabla u |^2 dx + \frac{1}{4} \int_{\Omega}^{} | \nabla u |^2 dx \\
            &\geq \frac{1}{4} \int_{\Omega}^{} | \nabla u |^2 dx + \frac{C}{4} \int_{\Omega}^{} | u |^2 dx \quad \parbox{5cm}{by Poincar\'{e}} \\
            &\geq C' \|u\|_{H^1_0(\Omega)}^2
          \end{align*}
          for some $C'>0$. Therefore, $F$ is coercive. This tells us that $F$ is bounded below and $F(u) \to \infty$ as $\|u\|_{H^1_0(\Omega)} \to
          \infty$.

          Let $I = \inf \{ F(u) : u \in \mathcal{A} \}.$ Pick a sequence $\{u_j\} \subset \mathcal{A}$ such that $F(u_j) \leq I+1$ and $F(u_j) \to I$.
          Write $u_j = v_j + g$ with $v_j \in H^1_0(\Omega)$. Then
          \begin{align*}
          F(v_j) &= \frac{1}{2} \int_{\Omega}^{} | \nabla v_j |^2 dx \\
          &\leq \int_{\Omega}^{} | \nabla g + \nabla v_j |^2 + | \nabla g |^2 dx \quad \parbox{5cm}{using part (a)} \\
          &= \int_{\Omega}^{} | \nabla u_j |^2 + | \nabla g |^2 dx \\
          &\leq I + 1 + C
          \end{align*}
          for some $C$ because $g \in H^1_0(\Omega)$. Because $I > -\infty$ and $F(u) \to \infty$ as $\|u\|_{H_0^1(\Omega)} \to \infty$, we know $\{v_j\}$ is bounded in $H^1_0(\Omega)$.


          $H_0^1 \subset L^2$ is a compact embedding, so there is a subsequence $v_{j_k} \to v$ in $L^2(\Omega)$. We can also find a further
          subsequence such that $v_{j_k} \rightharpoonup \tilde{v}$ in $H^1_0(\Omega)$ (we denote this further subsequence as $v_{j_k}$ as well).
          Because $\tilde{v} \in H^1_0(\Omega) \subset L^2(\Omega)$, we must have $v = \tilde{v}$.

          By problem 2, we know
          \[ \|v\|_{H_0^1(\Omega)} \leq \liminf_{k \to \infty} \|v_{j_k}\|_{H^1_0(\Omega)} .\]

          Let $u = v + g$. Then,
          \begin{align*}
            F(u) &= F(v+g) \\
            &= \frac{1}{2} \int_{\Omega}^{} | \nabla (v+g) |^2 dx \\
            &= \frac{1}{2} \|v+g\|_{H^1_0}^2 - \frac{1}{2} \|v+g\|_{L^2}^2 \\
            &\leq \liminf_{k \to \infty} \left( \frac{1}{2} \|v_{j_k}+g\|_{H^1_0}^2 - \frac{1}{2} \|v_{j_k}+g\|_{L^2}^2 \right) \quad \parbox{5cm}{by weak
            convergence in $H^1_0$ and strong convergence in $L^2$} \\
            &= \liminf_{k \to \infty} F(v_{j_k}+g) \\
            &= \liminf_{k \to \infty} F(u_{j_k}) \\
            &= I
          \end{align*}

          Thus, we have existence of a minimizer $u \in \mathcal{A}$.

      \end{enumerate}

    \end{solution}

  \item McOwen 7.1.5
    \begin{problem}
      Suppose we want to solve $\Delta u = 0$ in $\Omega$ with $\partial u / \partial \nu = h$ on $\partial \Omega$, where $\Omega$ is a smooth
      bounded domain in $\R^n$ and $h \in C^\infty(\partial \Omega)$ satisfies $\int_{\partial \Omega}^{} h dS = 0$. Show how to obtain a weak
      solution $u \in H^{1}(\Omega)$ from the existence theory discussed in this section.
    \end{problem}

    \begin{solution}
      Multiplying by a test function $v \in H^1(\Omega)$, we get the weak formulation
      \[ \int_{\Omega}^{} \nabla u \cdot \nabla v dx = \int_{\partial \Omega}^{} v h dS .\]

      Define
      \[ \mathcal{A} = \{ u \in H^1(\Omega) : (\partial u / \partial \nu - h) \big|_{\partial \Omega} = 0, \int_{\Omega}^{} u dx = 0 \} .\]
      We consider the functional
      \[ F(u) = \frac{1}{2} \int_{\Omega}^{} | \nabla u|^2 dx .\]

      We will show existence of a minimizer $u$ of $F$ in $\mathcal{A}$. Let
      \[ I = \inf \{ F(u) : u \in \mathcal{A} \} .\]

      For any $u$, $F(u) \geq 0$, we $I \geq 0$. Take a sequence $\{ u_j \}_{j=1}^\infty \subset \mathcal{A}$ such that $F(u_j) \to I$ and $F(u_j) \leq I + 1$ for all
      $j$.

      Write $u_j = v_j + h$ (where we have extended $h$ to be in $H^1(\Omega)$ in such a way that $h$ has average zero). Then
      $\int_{\Omega}^{} v_j dx = 0$ for all $j$, and $\partial v / \partial \nu = 0$ on $\partial \Omega$.

      For any $w \in H^1(\Omega)$ such that $\int_{\Omega}^{} w dx = 0$,
      \begin{align*}
        F(w) &= \int_{\Omega}^{} |\nabla w|^2 dx \\
        &= \frac{1}{2} \int_{\Omega}^{} |\nabla w|^2 dx + \frac{1}{2} \int_{\Omega}^{} |\nabla w|^2 dx \\
        &\geq \frac{1}{2} \int_{\Omega}^{} |\nabla w|^2 dx + \frac{C}{2} \int_{\Omega}^{} |w|^2 dx \quad \parbox{5cm}{by Poincar\'{e}} \\
        &\geq C' \|w\|_{H^1(\Omega)}^2
      \end{align*}
      Therefore, $F$ is coercive. So $F(u) \to \infty$ as $\|u\|_{H^1} \to \infty$, and $\{u_j\}_{j=1}^\infty$ is a bounded sequence in $H^1(\Omega)$.
      Then
      \begin{align*}
        F(v_j) &= \frac{1}{2} \int_{\Omega}^{} |\nabla v_j|^2 dx \\
        &\leq \int_{\Omega}^{} | \nabla h + \nabla v_j |^2 + | \nabla h |^2 dx \quad \parbox{5cm}{using part (a) of Problem 3} \\
        &= \int_{\Omega}^{} |\nabla u_j|^2 + |\nabla h|^2 dx \\
        &\leq I+1+C
      \end{align*}
      for some constant $C$. Thus $\{v_j\}_{j=1}^\infty$ is bounded in $H^1(\Omega)$ as well. Becuase $H^1 \subset L^2$ is a compact embedding, we can
      pass to a subsequence to get $v_j \to v$ for some $v \in L^2(\Omega)$. By passing to a further subsequence, we get $v_j \rightharpoonup
      \tilde{v}$ for some $\tilde{v} \in H^1(\Omega)$. Because $\tilde{v} \in H^1(\Omega) \subset L^2(\Omega)$, we know $\tilde{v} = v$.

      Let $u = v + h$. Then
      \begin{align*}
        F(u) &= F(v+h) \\
        &= \frac{1}{2} \int_{\Omega}^{} | \nabla(v + h) |^2 dx \\
        &= \frac{1}{2} \|v+h\|_{H^1}^2 - \frac{1}{2} \|v+h\|_{L^2}^2 \\
        &\leq \liminf_{j \to \infty} \left( \frac{1}{2} \|v_j + h\|_{H^1}^2 - \frac{1}{2} \|v_j + h\|_{L^2}^2 \right) \\
        &= \liminf_{j \to \infty} F(v_j + h) \\
        &= \liminf_{j \to \infty} F(u_j) \\
        &= I
      \end{align*}

      Therefore, $u$ minimizes $F$ over $\mathcal{A}$.

    \end{solution}

    \pagebreak

  \item McOwen 7.1.6
    \begin{problem}
      Suppose $\Omega$ is a bounded domain, $q(x)$ is a bounded function on $\Omega$ satisfying $q(x) \leq \eta$, and $f \in L^2(\Omega)$. If $\eta
      \geq 0$ is sufficiently small, then show that the Dirichlet problem $\Delta u + q(x) u = f$ in $\Omega, u = 0$ on $\partial \Omega$ admits a
      weak solution.
    \end{problem}

    \begin{solution}

      Let $v \in H_0^1(\Omega)$ and $u$ satisfy the given equation. Then
      \begin{align*}
        \int_{\Omega}^{} f v dx &= \int_{\Omega}^{} \Delta u v + q(x) u v dx \\
        &= \int_{\Omega}^{} - \nabla u \cdot \nabla v + q(x)u v dx + \int_{\partial \Omega}^{} \frac{\partial u}{\partial \nu} v dS \\
        &= \int_{\Omega}^{} - \nabla u \cdot \nabla v + q(x) u v dx
      \end{align*}

      Therefore, a weak solution satisfies
      \[ 0 = \int_{\Omega}^{} \nabla u \cdot \nabla v - q(x) u v + f v dx \]
      for all $v \in H^1_0(\Omega)$.
    \end{solution}

    Define the functional
    \[ F(u) = \int_{\Omega}^{} \frac{1}{2} | \nabla u |^2 - \frac{1}{2} q(x) |u|^2 + f u dx .\]

    Let $u, v \in H^1_0(\Omega)$. Then
    \begin{align*}
      F'(u) v &= \lim_{\varepsilon \to 0} \frac{1}{\varepsilon} \int_{\Omega}^{} \frac{1}{2} |\nabla(u+\varepsilon v)|^2 - \frac{1}{2} |\nabla u|^2 -
      \frac{1}{2} q(x) |u + \varepsilon v|^2 + \frac{1}{2} q(x) |u|^2 + f(u+ \varepsilon v) - f u dx \\
      &= \lim_{\varepsilon \to 0} \frac{1}{\varepsilon} \int_{\Omega}^{} \varepsilon \nabla u \cdot \nabla v + \frac{\varepsilon^2}{2} |\nabla v|^2 -
      \varepsilon q(x) u v - \frac{\varepsilon^2}{2} q(x) |v|^2 + \varepsilon f v dx \\
      &= \int_{\Omega}^{} \nabla u \cdot \nabla v - q(x) uv + fv dx
    \end{align*}

    From here, we see critical points of $F$ will be weak solutions of the given equation.

    Also,
    \begin{align*}
      | F(u+v) - F(u) - F'(u)v | &= \int_{\Omega}^{} \frac{1}{2} | \nabla (u+v) |^2 - \frac{1}{2} q(x) |u+v|^2 + f(u+v) \\
      &\quad \quad - \frac{1}{2} |\nabla u|^2 + \frac{1}{2} q(x) |u|^2 - f u - \nabla u \cdot \nabla v + q(x) uv - fv dx \\
      &= \int_{\Omega}^{} \frac{1}{2} ( | \nabla u |^2 + 2 \nabla u \cdot \nabla v + | \nabla v |^2 ) - \frac{1}{2} q(x) ( |u|^2 + 2uv + |v|^2) + f(u + v) \\
      &\quad \quad \frac{1}{2} |\nabla u|^2 + \frac{1}{2} q(x) |u|^2 - fu - \nabla u \cdot \nabla v + q(x) uv - fv dx \\
      &= \int_{\Omega}^{} \frac{1}{2} |\nabla v|^2 - q(x) |v|^2 dx \\
      &\leq C \|v\|_{H_0^1(\Omega)}^2 \quad \parbox{5cm}{because $q(x)$ is bounded}
    \end{align*}

    Therefore, $F$ is differentiable with derivative as given.

    Now we will show $F$ is coercive.
    \begin{align*}
      \left| \int_{\Omega}^{} fu dx \right| &\leq \|f\|_{L^2}^2 \|u\|_{L^2}^2 \\
      &\leq \frac{1}{2 \varepsilon} \|f\|_{L^2}^2 + \frac{\varepsilon}{2} \|u\|_{L^2}^2
    \end{align*}

    Then
    \begin{align*}
      F(u) &\geq \int_{\Omega}^{} \frac{1}{2} | \nabla u|^2 - \frac{\eta}{2} \|u\|_{L^2}^2 - \frac{1}{2 \varepsilon} \|f\|_{L^2}^2 -
      \frac{\varepsilon}{2} \|u\|_{L^2}^2 \\
      &\geq \left( \frac{1}{2} - \frac{C \eta}{2} - \frac{C \varepsilon}{2} \right) \|\nabla u\|_{L^2}^2 - \frac{1}{2\varepsilon} \|f\|_{L^2}^2
      \parbox{5cm}{by Poincar\'{e}} \\
      &\geq \tilde{C} \|\nabla u\|_{L^2}^2 + \hat{C} \|f\|_{L^2}^2
    \end{align*}
    for some $\tilde{C} >0$ if $C \eta < 1$ by taking $\varepsilon$ sufficiently small. By again using Poincar\'{e}, we get
    \[ F(u) \geq C \|u\|_{H^1_0}^2 + \hat{C} \|f\|_{L^2}^2 \]
    for $\eta$ sufficiently small. Thus, $F$ is coercive.

    Now we must show $F$ attains its minimum. Let
    \[ I = \inf \{ F(u) | u\in H^1_0(\Omega) \} .\]
    Take a sequence $\{u_j\} \subset H_0^1(\Omega)$ such that $F(u_j) \to I$ as $j \to \infty$ and $F(u_j) \leq I+1$ for all $j$. By the coercivity,
    $\{ u_j \}$ is bounded in $H_0^1(\Omega)$. Because $H^1_0 \subset L^2$ is a compact embedding, by passing to a subsequence, we get
    \[ u_j \to u \text{ in } L^2 .\]
    Because our sequence is bounded, we can take a further subsequence to find
    \[ u_j \rightharpoonup \tilde{u} \text{ in } H^1_0 .\]
    Because $\tilde{u} \in H^1_0(\Omega) \subset L^2(\Omega)$, we must have $u = \tilde{u}$.

    Because $f, u_j, u \in L^2(\Omega)$,
    \begin{align*}
      \left| \int_{\Omega}^{} f u_j dx - \int_{\Omega}^{} fu dx \right| &= \left| \int_{\Omega}^{} f (u_j - j)dx \right| \\
      &\leq \|f\|_{L^2} \|u_j - u\|_{L^2} \\
      &\quad \to 0 \text{ as } j \to \infty
    \end{align*}

    Similarly, because $|q(x)| \leq \eta$,
    \begin{align*}
      \left| \int_{\Omega}^{} q(x) |u_j|^2 dx - \int_{\Omega}^{} q(x) |u|^2 dx \right| &= \left| \int_{\Omega}^{} q(x) \left( |u_j|^2 - |u|^2 \right)
      dx \right| \\
      &\leq \eta \|u_j - u\|_{L^2}^2 \\
      &\quad \to 0 \text{ as } j \to \infty
    \end{align*}

    Now we check
    \begin{align*}
      F(u) &= \int_{\Omega}^{} \frac{1}{2} |\nabla u|^2 - q(x) |u|^2 + fu dx \\
      &= \|u\|_{H_0^1}^2 - \|u\|_{L^2}^2 - \int_{\Omega}^{} q(x) |u|^2 + fu dx \\
      &\leq \liminf_{j \to \infty} \|u_j\|_{H^1_0}^2 - \|u_j\|_{L^2}^2 - \int_{\Omega}^{} q(x) |u_j|^2 + fu_j dx \\
      &= \liminf_{j \to \infty} F(u_j) \\
      &= I
    \end{align*}
    where we have used the strong convergence in $L^2$, weak convergence in $H^!_0$, and the weak lower-semicontinuity of norms.
    Therefore, $F(u) = I$, and $u \in H^1_0(\Omega)$ is a minimizer.

  \item McOwen 7.1.9
    \begin{problem}
      Let $\Omega$ be a smooth bounded domain in $\R^n, g \in H^{2}(\Omega)$, and $h \in H^{1}(\Omega)$. Show that a critical point of the
      functional
      \[ F(u) = \frac{1}{2} \int_{\Omega}^{} (\Delta u)^2 dx dy \]
      for $u \in \mathcal{A} \equiv \{ u \in H^{2}(\Omega) : u - g \in H^{2}_0(\Omega) \text{ and } \partial u / \partial \nu - h \in H_0^{1}
      (\Omega) \}$ is a weak solution of the biharmonic equation $\Delta^2 u = \Delta (\Delta u) = 0$ in $\Omega$, with Dirichlet boundary conditions
      $u = g$ and $\partial u / \partial \nu = h$ on $\partial \Omega$.
    \end{problem}

    \begin{solution}
      Take $v \in H^2_0(\Omega)$ with $\partial_{x_i} v \in H^1_0(\Omega)$ for all $i$. Then
      \begin{align*}
        \int_{\Omega}^{} \Delta (\Delta u) v dx &= - \int_{\Omega}^{} \nabla (\Delta u) \cdot \nabla v dx \\
        &= \int_{\Omega}^{} \Delta u \Delta v dx
      \end{align*}
      This gives us our weak formulation for the biharmonic equation.

      Take $u \in \mathcal{A}$. Then for any $v \in H^1_0(\Omega)$, we have
      \begin{align*}
        F'(u)v &= \lim_{\varepsilon \to 0} \frac{1}{2 \varepsilon} \int_{\Omega}^{} ( \Delta(u + \varepsilon v ))^2 - (\Delta u)^2 dx \\
        &= \lim_{\varepsilon \to 0} \frac{1}{2 \varepsilon} \int_{\Omega}^{} 2 \varepsilon \Delta u \Delta v + \varepsilon^2 (\Delta v)^2 dx \\
        &= \int_{\Omega}^{} \Delta u \Delta v dx
      \end{align*}

      Thus critical points of the given functional for $u \in \mathcal{A}$ are weak solutions of the biharmonic equation.
    \end{solution}

    \pagebreak

  \item McOwen 7.2.2
    \begin{problem}
      In Theorem 2 of this section, we have not actually verified that every eigenvalue of
      \[ \begin{cases}
          \Delta u + \lambda u = 0 &\text{in } \Omega \\
          u = 0 &\text{on } \partial \Omega
      \end{cases} \]
      is one of the $\lambda_n$ generated by the functional $F$. Use the completeness of the eigenfuctions to prove this fact.
    \end{problem}

    \begin{solution}

      Let $\{\lambda_i\}_{i=1}^\infty$ be the eigenvalues generated by the functional $F$ with associated eigenfunctions $\{u_i\}_{i=1}^\infty$. Assume $u \not = 0$ satisfies
      \[ \begin{cases}
          \Delta u + \lambda u = 0 &\text{in } \Omega \\
          u = 0 &\text{on } \partial \Omega
      \end{cases} \]
      where $\lambda \not = \lambda_i$ for all $i$. Because the eigenfunctions form an orthonormal basis for $L^2$, we can expand
      \[ u = \sum_i \alpha_i u_i \]
      where $\alpha_i = \la u, u_i \ra .$ Then for any $i$,
      \begin{align*}
        -\lambda \alpha_i &= \la -\lambda u, u_i \ra \\
        &= \la \Delta u , u_i \ra \\
        &= \la u, \Delta u_i \ra \\
        &= \la u, -\lambda_i u_i \ra \\
        &= - \lambda_i \alpha_i
      \end{align*}

      Therefore,
      \[ (\lambda - \lambda_i) \alpha_i = 0 .\]
      Because $\lambda \not = \lambda_i$ for all $i$, we know $\alpha_i = 0$ for all $i$. Therefore, $u = 0$, which is a contradiction. Thus
      $\{\lambda_i\}_{i=1}^\infty$ contains all of the eigenvalues for the Laplacian.
    \end{solution}

  \item McOwen 7.2.3
    \begin{problem}
      Let $\lambda_1, \lambda_2, \dots$ be eigenvalues of the Laplacian with corresponding eigenfunctions $u_1, u_2, \dots$. Let $f \in
      H^{1}_0(\Omega)$ and $ \alpha_n = \la f, u_n \ra$. Then
      \[ \| \nabla f \|_2^2 = \sum_{n = 1}^\infty \alpha_n^2 \lambda_n .\]
    \end{problem}

    \begin{solution}
      Following the same argument used to show $f = \sum_{i=1}^\infty \alpha_i u_i$ where $\alpha_i = \la f, u_i \ra$, we can show
      \[ \nabla f = \sum_{i=1}^\infty \alpha_i \nabla u_i .\]

      Then
      \begin{align*}
        \|\nabla f\|_{L^2}^2 &= \la \sum_{i=1}^\infty \alpha_i \nabla u_i, \sum_{i=1}^\infty \alpha_i \nabla u_i \ra \\
        &= \sum_{i,j=1}^\infty \alpha_i \alpha_j \la \nabla u_i, \nabla u_j \ra \\
        &= \sum_{i=1}^\infty \alpha_i^2 \la \nabla u_i, \nabla u_i \ra \\
        &= \sum_{i=1}^\infty \alpha_i^2 \lambda_i
      \end{align*}
    \end{solution}

  \item McOwen 7.2.4
    \begin{problem}
      Consider a uniformly elliptic operator $L$ in divergence form, and let us pose the eigenvalue problem
      \begin{equation}
        \begin{cases}
          Lu + \lambda u = 0 &\text{in } \Omega \\
          u = 0 &\text{on } \partial \Omega
        \end{cases}
        \label{eqn:ell_div}
      \end{equation}

      \begin{enumerate}
        \item Give a variational formulation for the first eigenvalue $\lambda_1$. Observe that if $c(x) \leq c_0$, then $\lambda_1 > - c_0$.

        \item Show that the eigenvalues form a nondecreasing sequence $\lambda_1 \leq \lambda_2 \leq \dots$ tending to infinity.

        \item Show that each eigenvalue has a finite-dimensional eigenspace, and the collection of all eigenfunctions $u_n$ forms an orthonormal basis
          for $L^2( \Omega)$.
      \end{enumerate}

    \end{problem}

    \begin{solution}

      Let $Lu = \sum_{i,j=1}^n (a^{ij} u_{x_i})_{x_j} + cu$ where $a^{ij} = a^{ji}$ and $c(x) \leq 0$ for all $x \in \Omega$.

      \begin{enumerate}
        \item
          Weak solutions of equation \eqref{eqn:ell_div} satisfy
          \[ \int_{\Omega}^{} \sum_{i,j=1}^n a^{ij} u_{x_i} v_{x_j} - c u v dx = \lambda \int_{\Omega}^{} uv dx \]
          for all $v \in H^1_0(\Omega)$.

          We will work with the following inner product:
          \[ \la u, v \ra_X = \int_{\Omega}^{} \sum_{i,j=1}^n a^{ij} u_{x_i} v_{x_j} dx + \int_{\Omega}^{} uv dx \]
          for $u,v \in H^1_0(\Omega)$. $\la \cdot, \cdot \ra$ is symmetric because $a^{ij}=a^{ji}$, and this inner product is clearly linear. By
          uniform ellipticity
          \[ \int_{\Omega}^{} \sum_{i,j=1}^n a^{ij} u_{x_i} u_{x_j} dx \geq \theta \int_{\Omega}^{} | \nabla u |^2 dx = \theta \| \nabla u \|_{L^2}^2 .\]
          Therefore, our inner product is positive definite because $H^1_0(\Omega)$ is positive definite with the standard inner product. Also,
          because $a^{ij}$ is continuous and $\Omega$ is bounded,
          \[ \left| \int_{\Omega}^{} \sum_{i,j=1}^n a^{ij} u_{x_i} u_{x_j} dx \right| \leq C \| \nabla u \|_{L^2}^2 .\]
          Therefore, $\la \cdot , \cdot \ra_X$ defines an inner product on $H^1_0(\Omega)$. We will denote the norm it induces as $\| \cdot \|_X$ and
          notice our use of uniform ellipticity and boundedness of $a^{ij}$ have shown this norm is equivalent to the standard norm on $H^1_0(\Omega)$.

          Define the functionals
          \[ F(u) = \int_{\Omega}^{} \sum_{i,j=1}^n a^{ij} u_{x_i} u_{x_j} - cu^2 dx \]
          and
          \[ G(u) = \int_{\Omega}^{} u^2 dx - 1 = \|u\|_{L^2}^2 -1 .\]

          For $u,v \in H^1_0(\Omega)$, we have
          \begin{align*}
            F'(u)v &= \lim_{\varepsilon \to 0} \frac{1}{\varepsilon} \int_{\Omega}^{} \sum_{i,j=1}^n a^{ij} \left( (u+\varepsilon v)_{x_i} (u+\varepsilon
            v)_{x_j} - u_{x_i} u_{x_j} \right) - c \left( (u + \varepsilon v)^2 - u^2) \right) dx \\
            &= \lim_{\varepsilon \to 0} \frac{1}{\varepsilon} \int_{\Omega}^{} \sum_{i,j=1}^n ( \varepsilon u_{x_i} v_{x_j} + \varepsilon
            u_{x_j} v_{x_i} + \varepsilon^2 v_{x_i} v_{x_j}) - 2 \varepsilon c u v - \varepsilon^2 c v^2 dx \\
            &= \lim_{\varepsilon \to 0} \int_{\Omega}^{} \sum_{i,j=1}^n a^{ij} ( 2 \varepsilon u_{x_i} v_{x_j} + \varepsilon^2 v_{x_i} v_{x_j} )
            - 2 \varepsilon c u v - \varepsilon^2 c v^2 dx \quad \parbox{5cm}{because $a^{ij} = a^{ji}$} \\
            &= \int_{\Omega}^{} 2 \sum_{i,j=1}^n a^{ij} u_{x_i} v_{x_j} - 2 c u v dx
          \end{align*}
          and
          \begin{align*}
            G'(u)v &= \lim_{\varepsilon \to 0} \int_{\Omega}^{} (u+\varepsilon v)^2 - u^2 dx - 1 + 1 \\
            &= \lim_{\varepsilon \to 0} \int_{\Omega}^{} 2 \varepsilon uv + \varepsilon^2 v^2 dx \\
            &= \int_{\Omega}^{} 2 uv dx
          \end{align*}

          Now we must show $F$ and $G$ are differentiable.
          \begin{align*}
            |F(u+v) - F(u) - F'(u)v| &= \int_{\Omega}^{} \sum_{i,j=1}^n a^{ij} \left( (u+v)_{x_i} (u+v)_{x_j} - u_{x_i} u_{x_j} - 2
            u_{x_i} v_{x_j} \right) \\
            &\quad \quad \quad - c(u+v)^2 + cu^2 + 2c u v dx \\
            &= \int_{\Omega}^{} \sum_{i,j=1}^n a^{ij} v_{x_i} v_{x_j} - c v^2 dx \\
            &\leq C \| v \|_X^2 \quad \parbox{5cm}{because $c$ is continuous and $\Omega$ is bounded} \\
            &= o(\|v\|_X)
          \end{align*}

          \begin{align*}
            |G(u+v) - G(u) - G'(u)v| &= \int_{\Omega}^{} (u+v)^2 - u^2 - 2uv dx \\
            &= \int_{\Omega}^{} v^2 dx \\
            &= \|v\|_{L^2}^2 \\
            &\leq \|v\|_X^2 \\
            &= o(\|v\|_X)
          \end{align*}

          Therefore, $F$ and $G$ are differentiable. Next we will show $F$ and $G$ are $C^1$.
          Take $u,v,w \in H^1_0(\Omega)$. Then
          \begin{align*}
            | \left( F'(u) - F'(v) \right) w |^2 &= \left| \int_{\Omega}^{} 2 \sum_{i,j=1}^n a^{ij} (u_{x_i} - v_{x_i}) w_{x_j}  - 2 c(u-v)w dx \right|^2 \\
            &\leq C \left| \int_{\Omega}^{} \sum_{i=1}^n (u_{x_i} - v_{x_i}) \sum_{j=1}^n w_{x_j} - (u-v)w dx \right|^2 \quad \parbox{4cm}{because
            $a^{ij}, c$ are continuous and $\Omega$ is bounded} \\
            &\leq C \left( \sum_{i=1}^n \|u_{x_i} - v_{x_i}\|_{L^2}^2 \sum_{j=1}^n \|w_{x_j}\|_{L^2}^2  + \|u-v\|_{L^2}^2 \|w\|_{L^2}^2 \right) \\
            &\leq \tilde{C} \| \nabla (u-v) \|_{L^2}^2 \| \nabla w \|_{L^2}^2 \quad \parbox{5cm}{by Poincar\'{e}}
          \end{align*}

          Therefore using the equivalence of our norms,
          \[ | (F'(u) - F'(v) ) w | \leq C \|u-v\|_X \|w\|_X .\]
          By definition, we have
          \begin{align*}
            \|F'(u) - F'(v)\|_{H^{-1}} &= \sup_{w \not = 0} \frac{| (F'(u) - F'(v)) w|}{\|w\|_X} \\
            &\leq C \|u-v\|_X
          \end{align*}

          Thus, $F$ is $C^1$.

          Also,
          \begin{align*}
            |(G'(u) - G'(v))w| &= \left| \int_{\Omega}^{} 2 (u-v) w dx \right| \\
            &\leq 2 \|u-v\|_{L^2} \|w\|_{L^2} \\
            &\leq 2 \|u-v\|_X \|w\|_X
          \end{align*}

          Therefore,
          \begin{align*}
            \|G'(u) - G'(v)\|_{H^{-1}} &= \sup_{w \not = 0} \frac{|(G'(u) - G'(v))w|}{\|w\|_X} \\
            &\leq \|u-v\|_X
          \end{align*}
          and $G$ is $C^1$ as well.

          Now we verify that $F$ attains its minimum on
          \[ \mathcal{C} = \{ u \in H^1_0(\Omega) | G(u)=0 \} .\]
          Let $I = \inf \{ F(u) | u \in \mathcal{C} \}.$ Take a sequence $\{u_j\} \in \mathcal{C}$ such that $F(u_j) \to I$ and $F(u_j) \leq I+1$ for
          all $j$. By the equivalence of our norms on $H^1_0(\Omega)$, $F$ is coercive, so $\{u_j\}$ is bounded. Therefore, by passing to
          subsequences, we have $u_j \to \bar{u}$ in $L^2(\Omega)$ and $u_j \rightharpoonup \tilde{u}$ in $H^1_0(\Omega)$. Because $L^2 \subset H^1_0,
          \bar{u} = \tilde{u}$.

          Because $u_j \to \bar{u}$ in $L^2$, we know
          \[ \|\bar{u}\|_{L^2} = \lim_{j \to \infty} \|u_j\|_{L^2} = 1 \]
          because $u_j \in \mathcal{C}$ for all $j$. Therefore, $\bar{u} \in \mathcal{C}$.
          Strong $L^2$ convergence also tells us that
          \begin{align*}
            \left| \int_{\Omega}^{} c \bar{u}^2 dx - \int_{\Omega}^{} c u_j^2 dx \right| &= \left| \int_{\Omega}^{} c (\bar{u}-u_j)^2 dx \right| \\
            &\leq C \|u_j - \bar{u} \|_{L^2}^2 \\
            &\quad \to 0 \text{ as } j \to \infty
          \end{align*}
          where we have used the continuity of $c$ and boundedness of $\Omega$ to conclude $|c(x)| \leq C$ for all $x \in \Omega$ and some $C>0$.
          Putting this all together, we get
          \begin{align*}
            F(\bar{u}) &= \|\bar{u}\|_{X}^2 - \|\bar{u}\|_{L^2}^2 - \int_{\Omega}^{} c \bar{u}^2 dx \\
            &\leq \liminf_{j \to \infty} \|u_j\|_{X}^2 - \|u_j\|_{L^2}^2 - \int_{\Omega}^{} c u_j^2 dx \quad \parbox{5cm}{by weak lower-semicontinuity
            of norms} \\
            &= \liminf_{j \to \infty} F(u_j) \\
            &= I
          \end{align*}

          Because $\bar{u} \in \mathcal{C}$, we know $F(\bar{u}) = I$, and $F$ attains its minimum on $\mathcal{C}$ at $\bar{u}$.

          Therefore, from the Euler-Lagrange equations, we know
          \[ \int_{\Omega}^{} \sum_{i,j=1}^n a^{ij} \bar{u}_{x_i} v_{x_j} - c u v dx = \lambda \int_{\Omega}^{} \bar{u} v dx \]
          for all $v \in H^1_0(\Omega)$, and $\bar{u}$ is an eigenfunction for $L$ corresponding to the eigenvalue $\lambda$ for some $\lambda$.
          Relabel $(\lambda, \bar{u})$ as $(\lambda_1, u_1)$.
          By taking $v = u_1$ above, we find
          \begin{align*}
            I &= F(u_1) \\
            &= \int_{\Omega}^{} \sum_{i,j=1}^n a^{ij} \partial_{x_i} u_1 \partial_{x_j} u_1 - c u_1^2 dx \\
            &= \lambda \int_{\Omega}^{} u^2 dx \\
            &= \lambda
          \end{align*}

          So we can define $\lambda_1$ by
          \[ \lambda_1 = \inf_{u \in H^1_0(\Omega)} \frac{\int_{\Omega}^{} \sum_{i,j=1}^n a^{ij} u_{x_i} u_{x_j} - cu^2 dx }{\int_{\Omega}^{} u^2 dx}
          .\]

          By uniform ellipticity, we know
          \[ \sum_{i,j}^n a^{ij} u_{x_i} u_{x_j} > \theta | \nabla u |^2 \geq 0 \]
          From here, we see that if $c(x) \leq c_0$ for all $x \in \Omega$, then
          \[ \lambda_1 \geq \frac{-c_0 \int_{\Omega}^{} u^2 dx}{\int_{\Omega}^{} u^2 dx} = -c_0 \]

        \item
          After finding $(\lambda_1, u_1)$, assume there is another eigenpair $(\lambda_2, u_2)$ with $\lambda_1 \not = \lambda_2$. Then
          \begin{align*}
            (\lambda_1 - \lambda_2) \int_{\Omega}^{} u_1 u_2 dx &= \int_{\Omega}^{} \lambda_1 u_1 u_2 dx - \int_{\Omega}^{} u_1 \lambda_2 u_2 dx \\
            &= \int_{\Omega}^{} \sum_{i,j=1}^n a^{ij} \partial_{x_i} u_1 \partial_{x_j} u_2 - c u_1 u_2 dx - \int_{\Omega}^{} \sum_{i,j=1}^n a^{ij}
            \partial_{x_i} u_2 \partial_{x_j} u_1 - c u_1 u_2 dx \\
            &= 0
          \end{align*}

          Because we assumed $\lambda_1 \not = \lambda_2$ we know $ \la u_1, u_2 \ra_{L^2} = 0$. So eigenfunctions corresponding to different
          eigenvalues are orthogonal in $L^2$. Therefore, when searching for $\lambda_2$ we should look in
          \[ X_1 = \{ v \in H^1_0 | \la v, u_1 \ra = 0 \} .\]
          From this we define
          \[ \lambda_2 = \inf \{ F(v) | v \in X_1 \text{ and } \|v\|_{L^2} = 1 \} = \inf_{v \in X_1} \frac{\int_{\Omega}^{} \sum_{i,j=1}^n a^{ij}
          u_{x_i} u_{x_j} - c u^2 dx }{\int_{\Omega}^{} u^2 dx} .\]

          Because $X_1 \subset H^1_0(\Omega)$, we see $\lambda_1 \leq \lambda_2$. By continuing inductively, we see that our eigenfunctions for
          subsequent eigenvalues will be orthonormal, and $\lambda_1 \leq \lambda_2 \leq \lambda_3 \leq \dots$

          Assume our eigenvalues are bounded. Because $c(x) \leq 0$ for all $x \in \Omega$, we have
          \begin{align*}
            \|u_n\|_X^2 &= \la u_n , u_n \ra_X \\
            &= \int_{\Omega}^{} \sum_{i,j=1}^n a^{ij} u_{x_i} u_{x_j} + u^2 dx \\
            &\leq \int_{\Omega}^{} \sum_{i,j=1}^n a^{ij} u_{x_i} u_{x_j} - c u^2 dx + \int_{\Omega}^{} u^2 dx \\
            &= \lambda_n + 1
          \end{align*}

          Because we assumed our eigenvalues are bounded, we have $\|u_n\|_X^2 \leq C$ for some $C$ and all $n$. So $\{u_j\}_{j=1}^\infty$ is a
          bounded sequence in $H^1_0(\Omega)$. Because $H^1_0$ is compactly embedded in $L^2$, we can find a subsequence $\{u_{j_k}\}$ converging in
          $L^2$. But by the Pythagorean Theorem,
          \[ \| u_i - u_k \|_{L^2}^2 = 2 \]
          for $i \not = j$ because our eigenfunctions are orthonormal, which is a contradiction. Therefore, $\lambda_n \to \infty$.

        \item
          Because $\lambda_n \to \infty$, each eigenvalue will only be found by our minimization procedure a finite number of times. For a fixed
          eigenvalue $\lambda_i$, this also produces an orthonormal set of eigenfunctions spanning the eigenspace corresponding to $\lambda_i$
          (if not, our minimization would have repeated $\lambda_i$ again). Therefore, each eigenspace is finite dimensional.
      \end{enumerate}

    \end{solution}

  \item McOwen Theorem 3, page 236
    \begin{problem}
      If $\widehat{\Omega} \subset \Omega$ are bounded domains, then $\lambda_n( \widehat{\Omega}) \geq \lambda_n (\Omega)$.
    \end{problem}

    \begin{solution}
      Take $u \in H^1_0(\hat{\Omega})$. Then there is a sequence $\{u_j\} \subset C_c^\infty(\hat{\Omega})$ such that $u_j \to u$ in
      $H^1(\hat{\Omega})$. If we define $\tilde{u}_j$ to equal $u_j$ on $\hat{\Omega}$ and to be 0 on $\Omega \setminus \hat{\Omega}$, we get
      $\tilde{u}_j \to \tilde{u}$ where $\tilde{u}$ is $u$ extended to be 0 on $\Omega \setminus \hat{\Omega}$. So we can consider
      $H^1_0(\hat{\Omega}) \subset H^1_0(\Omega)$.

      For $v_1, \dots, v_{n-1} \in H^1_0(\Omega)$, define
      \begin{align*}
        &m \{v_1, \dots, v_{n-1} \} = \\
        &\quad \inf_{u \in H^1_0(\Omega)} \left\{ \int_{\Omega}^{} | \nabla u|^2 dx : \|u\|_2 = 1, \la u, v_i \ra = 0 \text{ for } i=1, \dots, n-1
      \right\}
      \end{align*}
      and for $\hat{v}_1, \dots , \hat{v}_{n-1} \in H^1_0(\hat{\Omega})$, define
      \begin{align*}
        &\hat{m} \{\hat{v}_1, \dots, \hat{v}_{n-1} \} = \\
        &\quad \inf_{u \in H^1_0(\hat{\Omega}) } \left\{ \int_{\hat{\Omega}}^{} | \nabla u|^2 dx : \|u\|_2 = 1, \la u, \hat{v}_i \ra = 0 \text{ for } i=1, \dots, n-1
      \right\}
    \end{align*}

      By the above identification, for any $v_1, \dots, v_{n-1} \in H^1_0(\hat{\Omega})$, we have
      \[ m \{ v_1, \dots, v_n \} \leq \hat{m} \{ v_1, \dots, v_n \} .\]

      Therefore,
      \begin{align*}
        \lambda_n (\Omega) &= \sup_{v_1, \dots, v_{n-1} \in H^1_0(\Omega) } m \{ v_1, \dots, v_{n-1} \} \\
        &\leq \sup_{v_1, \dots , v_{n-1} \in H^1_0(\hat{\Omega})} \hat{m} \{ v_1, \dots, v_{n-1} \} \\
        &= \lambda_n (\hat{\Omega})
      \end{align*}

    \end{solution}

\end{enumerate}
\end{document}



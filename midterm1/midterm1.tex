%        File: midterm1.tex
%     Created: Sat Mar 26 12:00 PM 2016 C
% Last Change: Sat Mar 26 12:00 PM 2016 C
%

\documentclass[a4paper]{article}

\title{Math 8584 Midterm 1 }
\date{4/4/16}
\author{Trevor Steil}

\usepackage{amsmath}
\usepackage{amsthm}
\usepackage{amssymb}
\usepackage{esint}

\newtheorem{theorem}{Theorem}[section]
\newtheorem{corollary}{Corollary}[section]
\newtheorem{proposition}{Proposition}[section]
\newtheorem{lemma}{Lemma}[section]
\newtheorem{definition}{Definition}[section]

\newenvironment{solution}{\emph{Solution.}}{}
\newenvironment{claim}{\textbf{Claim.}}{}
\newenvironment{problem}{\textbf{Problem.}}{}

\newcommand{\R}{\mathbb{R}}
\newcommand{\N}{\mathbb{N}}
\newcommand{\C}{\mathbb{C}}
\newcommand{\supp}[1]{\mathop{\mathrm{supp}}\left(#1\right)}
\newcommand{\lip}[1]{\mathop{\mathrm{Lip}}\left(#1\right)}
\newcommand{\curl}{\mathrm{curl}}
\newcommand{\la}{\left \langle}
\newcommand{\ra}{\right \rangle}
\renewcommand{\vec}[1]{\mathbf{#1}}

\begin{document}
\maketitle

\begin{enumerate}
  \item
    \begin{problem}
      Construct a function $u \in S(\R)$ (i.e. a Schwartz-class function in one dimension) that is not exponentially small at infinity.
      That is, construct a Schwartz-class function which satisfies that for all $a > 0$,
      \[ e^{a |x|^a} u \not \in L^\infty(\R) .\]
    \end{problem}

    \begin{solution}
    \end{solution}

  \item
    \begin{problem}
      Justify the following inequalities from Hunter's notes for a smooth function $u$ on $\R^n$:
      \begin{align*}
        \Delta u(x) &= \lim_{r \to 0^+} \fint_{B_r(x)} \Delta u dx \\
        &= \lim_{r \to 0^+} \fint_{\partial B_r(x)}^{} u(y) d\sigma(y) \\
        &= \lim_{r \to 0^+} \frac{2n}{r^2} \left[ \fint_{\partial B_r(x)}^{} u(y) d\sigma(y) - u(x) \right]
      \end{align*}

    \end{problem}

    \begin{solution}
      Because $u$ is smooth, $\Delta u$ is smooth, and in particular, $\Delta u$ is continuous. By Lebesgue Differentiation,
      \[ \Delta u(x) = \lim_{r \to 0^+} \fint_{B_r(x)}^{} \Delta u dx .\]

      By the Divergence Theorem, we know
      \[ \int_{B_r(x)}^{} \Delta u dx = \int_{\partial B_r(x)}^{} \frac{\partial u}{\partial \eta}(z) d \sigma(z) ,\]
      where $\eta$ is the unit outer normal to $B_r(x)$. Now we introduce the change of variables $z = x+ry$. Then we have
      \[ \int_{\partial B_r(x)}^{} \frac{\partial u}{\partial \eta}(z) d \sigma(z) = r^{n-1} \int_{B_1(0)}^{} \frac{\partial u}{\partial r} d
      \sigma(y) .\]
      The region we are integrating over no longer depends on $r$, so we can move the derivative outside to get
      \[ \int_{\partial B_r(x)}^{} \frac{\partial u}{\partial \eta}(z) d \sigma(z) = r^{n-1} \frac{\partial}{\partial r} \int_{\partial B_1(0)}^{}
      u d \sigma .\]

      Therefore,
      \begin{align*}
        \fint_{B_r(x)}^{} \Delta u dx
        &=\frac{1}{m(B_r)} \int_{B_r(x)}^{} \Delta u dx \\
        &= \frac{r^{n-1}}{m(B_r)} \frac{\partial}{\partial r} \int_{\partial B_1(0)}^{} u d\sigma \\
        &= \frac{1}{r} \frac{\partial}{\partial r} \frac{1}{m(B_1)} \int_{\partial B_1(0)}^{} u d\sigma \\
        &= \frac{n}{r} \frac{\partial}{\partial r} \frac{1}{m(\partial B_1)} \int_{\partial B_1(0)}^{} u d\sigma \\
        &= \frac{n}{r} \frac{\partial}{\partial r} \frac{1}{r^{n-1} m(\partial B_1)} \int_{\partial B_r(x)}^{} u d\sigma \quad \parbox{4cm}{after
        rescaling in the opposite way as before} \\
        &= \frac{n}{r} \frac{\partial}{\partial r} \frac{1}{m(\partial B_r)} \int_{\partial B_r(x)}^{} u d\sigma \\
        &= \frac{n}{r} \frac{\partial}{\partial r} \fint_{\partial B_r(x)}^{} u d\sigma
      \end{align*}

    \end{solution}

  \item Evans, Chapter 6 \#7 \\
    \begin{claim}
      Let $u \in H^1(\R^n)$ have compact support and be a weak solution of the semilinear PDE
      \[ - \Delta u + c(u) = f \quad \text{in } \R^n ,\]
      where $f \in L^2(\R^n)$ and $c: \R \to \R$ is smooth, with $c(0) = 0$ and $c' \geq 0.$ Then $u \in H^2(\R^n)$.
    \end{claim}

    \begin{proof}
      Let $U \subset \R^n$ be compact such that $\supp{u} \subset U$. Because $u$ is a weak solution of
      \[ -\Delta u + c(u) = f ,\]
      we know
      \[ \sum_{i = 1}^n \int_{U}^{} u_{x_i} v_{x_i} dx = \int_{U}^{} (f - c(u)) v dx \]
      for all $v \in H_0^1(U)$.

      Let $|h| > 0$ be small, $k \in \{1,2,\dots,n\}$, and choose
      \[ v = -D_k^{-h} (D_k^h u) ,\]
      where $D_k^h$ is the difference quotient
      \[ D_k^h u(x) = \frac{u(x + h e_k) - u(x)}{h} .\]

      We will let
      \[ A = \sum_{i=1}^n \int_{U}^{} u_{x_i} v_{x_i} dx \]
      and
      \[ B = \int_{U}^{} (f - c(u)) v dx .\]
      First, we will look at $A$.
      We have integration by parts and product formulas for difference quotients given as
      \[ \int_{U}^{} v D_k^{-h} w dx = - \int_{U}^{} w D_k^h v dx ,\]
      and
      \[ D_k^h (vw) = v^h D_k^h w + w D_k^h v ,\]
      where $v^h(x) := v(x + he_k)$.
      So we have
      \begin{align*}
        A &= - \sum_{i=1}^n \int_{U}^{} u_{x_i} \left[ D_k^{-h} (D_k^h u) \right]_{x_i} dx \\
        &= \sum_{i=1}^n \int_{U}^{} D_k^h u_{x_i} \left( D_k^h u \right)_{x_i} dx \quad \parbox{6cm}{NEED TO KNOW PARTIAL DERIVATIVE COMMUTES WITH
        DIFFERENCE QUOTIENT} \\
        &= \sum_{i=1}^n \int_{U}^{} D_k^h u_{x_i} D_k^h u_{x_i} dx \\
        & = \| D (D_k^h u) \|_{L^2(U)}^2 \\
        &= \| D_k^h Du \|_{L^2(U)}^2
      \end{align*}

      Now we turn to $B$.
      By Theorem 5.8.2 in Evans,
      \begin{align*}
        \int_{U}^{} |v|^2 dx &= \int_{U}^{} |D_k^{-h} (D_k^h u)|^2 dx \\
        & \leq C \int_{U}^{} |D (D_k^h u)|^2 dx \\
        &= C \int_{U}^{} |D_k^h Du|^2 dx \quad \parbox{6cm}{NEED TO CHECK D AND $D_k^h$ COMMUTE}
      \end{align*}

      Therefore, by using Young's inequality we get that for any $\varepsilon > 0$,
      \begin{align*}
        |B| &= \int_{U}^{} (|f| + |c(u)|) |v| dx \\
        &\leq \int_{U}^{} \frac{|f|^2}{2 \varepsilon} + \frac{|c(u)|^2}{2 \varepsilon} + \varepsilon |v|^2 dx \\
        &\leq \int_{U}^{} \frac{|f|^2}{2 \varepsilon} + \frac{|c(u)|^2}{2 \varepsilon} + C\varepsilon |D_k^h Du|^2 dx
      \end{align*}

      Take $\varepsilon = \frac{1}{2C}$. Then combining $A$ and $B$, we have
      \begin{align*}
        \| D_k^h Du \|_{L^2(U)}^2 &= |A| \\
        &= |B| \\
        &\leq \int_{U}^{} C (|f|^2 + |c(u)|^2)dx + \frac{1}{2} \| D_k^h Du \|_{L^2(U)}^2
      \end{align*}
      Thus,
      \[ \| D_k^h Du \|_{L^2(U)}^2 \leq 2C \int_{U}^{} |f|^2 + |c(u)|^2 dx .\]

      For any $x \in U$, we have
      \begin{align*}
        |c(u(x))| &= \int_{0}^{u(x)} c'(s) ds \quad \parbox{5cm}{because $c(0)=0$} \\
        &\leq \| c' \|_{L^\infty} |u(x)|
      \end{align*} <++>
    \end{proof}

  \item Evans, Chapter 6 \#8 \\
    \begin{claim}
      Let $u$ be a smooth solution of the uniformly elliptic equation $Lu = - \sum_{i,j=1}^n a^{ij}(x) u_{x_i x_j} = 0$ in $U$. Assume that the
      coefficients have bounded derivatives.
      Set $v := |Du|^2 + \lambda u^2$. Then
      \[ Lv \leq 0 \quad \text{in } U \]
      if $\lambda$ is large enough. Therefore,
      \[ \| Du \|_{L^\infty(U)} \leq C \left( \| Du \|_{L^\infty(\partial U)} + \| u \|_{L^\infty(\partial U)} \right). \]
    \end{claim}

    \begin{proof}
    \end{proof}

  \item Evans, Chapter 6 \#10 \\
    \begin{problem}
      Assume $U$ is connected. Use (a) energy methods and (b) the maximum principle to show that the only smooth solutions of the Neumann
      boundary-value problem
      \[ \begin{cases}
          - \Delta u = 0 &\text{in } U \\
          \frac{\partial u}{\partial \nu} = 0 &\text{on } \partial U
      \end{cases} \]
      are $u \equiv C$, for some constant $C$.
    \end{problem}

      \begin{proof}
      \end{proof}

  \item Renardy-Rogers, pg. 225 \#5 \\
    \begin{claim}
      Assume $u,v \in H^1(\R)$. Then $uv \in H^1(\R)$.
    \end{claim}

    \begin{proof}
    \end{proof}

  \item Renardy-Rogers, pg. 225 \#8 \\
    \begin{claim}
      Let $\Omega$ satisfy the assumptions of Theorem 7.29. Then every weakly convergent sequence in $H^{k+1}(\Omega)$ converges strongly in
      $H^k(\Omega)$.
    \end{claim}

    \begin{proof}
    \end{proof}

  \item Renardy-Rogers, pg. 225 \#10 \\
    \begin{claim}
      Let $\Omega$ be bounded (no assumptions on $\partial \Omega$). Then $H_0^{k+1}(\Omega)$ is compactly embedded in $H^k(\Omega)$.
    \end{claim}

    \begin{proof}
    \end{proof}

\end{enumerate}
\end{document}



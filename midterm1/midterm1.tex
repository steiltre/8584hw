%        File: midterm1.tex
%     Created: Sat Mar 26 12:00 PM 2016 C
% Last Change: Sat Mar 26 12:00 PM 2016 C
%

\documentclass[a4paper]{article}

\title{Math 8584 Midterm 1 }
\date{4/4/16}
\author{Trevor Steil}

\usepackage{amsmath}
\usepackage{amsthm}
\usepackage{amssymb}
\usepackage{esint}

\newtheorem{theorem}{Theorem}[section]
\newtheorem{corollary}{Corollary}[section]
\newtheorem{proposition}{Proposition}[section]
\newtheorem{lemma}{Lemma}[section]
\newtheorem{definition}{Definition}[section]

\newenvironment{solution}{\emph{Solution.}}{}
\newenvironment{claim}{\textbf{Claim.}}{}
\newenvironment{problem}{\textbf{Problem.}}{}

\newcommand{\R}{\mathbb{R}}
\newcommand{\N}{\mathbb{N}}
\newcommand{\C}{\mathbb{C}}
\newcommand{\supp}[1]{\mathop{\mathrm{supp}}\left(#1\right)}
\newcommand{\lip}[1]{\mathop{\mathrm{Lip}}\left(#1\right)}
\newcommand{\curl}{\mathrm{curl}}
\newcommand{\la}{\left \langle}
\newcommand{\ra}{\right \rangle}
\renewcommand{\vec}[1]{\mathbf{#1}}

\begin{document}
\maketitle

\begin{enumerate}
  \item
    \begin{problem}
      Construct a function $u \in S(\R)$ (i.e. a Schwartz-class function in one dimension) that is not exponentially small at infinity.
      That is, construct a Schwartz-class function which satisfies that for all $a > 0$,
      \[ e^{a |x|^a} u \not \in L^\infty(\R) .\]
    \end{problem}

    \begin{solution}
      Let $u(x) = x^{- \ln x}$. Then
      \begin{align*}
        u(x) e^{a |x|^a} &= e^{-(\ln x)^2} e^{a |x|^a} \\
        &= e^{a |x|^a - (\ln x)^2}
      \end{align*}
      We would like to show that $a |x|^a - (\ln x)^2 \to \infty$ as $x \to \infty$ in order to show $u e^{a |x|^a} \not\in L^\infty(\R)$.
      To do this, we make the substitution $x = e^t$ to get
      \begin{align*}
        \lim_{x \to \infty} a |x|^a - (\ln x)^2 &= \lim_{t \to \infty} a e^{ta} - t^2 \\
        &= \infty
      \end{align*}
      for any value of $a > 0$. Therefore,
      \begin{align*}
        \lim_{x \to \infty} u(x) e^{a |x|^a} &= \lim_{x \to \infty} e^{a |x|^a - (\ln x)^2} \\
        &= e^{ \lim_{x \to \infty}  a|x|^a - (\ln x)^2} \\
        &= \infty
      \end{align*}
      Thus, $u(x) e^{a|x|^a} \not\in L^{\infty}(R)$ for any value of $a > 0$, and $u(x)$ is therefore subexponential. Now we must show $u$ is Schwartz.
      Take $k > 0$. Then
      \begin{align*}
        \lim_{x \to \infty} x^k u(x) &= \lim_{x \to \infty} x^{k - \ln x} \\
        &= 0
      \end{align*}

      Through logarithmic differentiation, we get
      \[ u'(x) = -2 x^{-\ln x - 1} \ln x.\]
      Then
      \begin{align*}
        \lim_{x \to \infty} x^k u'(x) &= - \lim_{x \to \infty} 2 x^{-\ln x + k -1} \ln x \\
        &= -2 \lim_{x \to \infty} e^{- (\ln x)^2 + (k -1)\ln x} e^{\ln(\ln x)} \\
        &= -2 \lim_{x \to \infty} e^{-(\ln x)^2 + (k-1) \ln x + \ln( \ln x)}
      \end{align*}

      To complete this computation, we make the substitution $y = \ln x$. Then
      \begin{align*}
        \lim_{x \to \infty} x^k u'(x) &= -2 \lim_{y \to \infty} e^{-y^2 + (k-1)y + \ln y} \\
        &= -2 \lim_{y \to \infty} y e^{-y^2 + (k-1)y} \\
        &= 0
      \end{align*}

      Taking further derivatives, similar calculations show $u(x) \in S(R)$.
    \end{solution}

  \item
    \begin{problem}
      Justify the following inequalities from Hunter's notes for a smooth function $u$ on $\R^n$:
      \begin{align*}
        \Delta u(x) &= \lim_{r \to 0^+} \fint_{B_r(x)} \Delta u dx \\
        &= \lim_{r \to 0^+} \fint_{\partial B_r(x)}^{} u(y) d\sigma(y) \\
        &= \lim_{r \to 0^+} \frac{2n}{r^2} \left[ \fint_{\partial B_r(x)}^{} u(y) d\sigma(y) - u(x) \right]
      \end{align*}

    \end{problem}

    \begin{solution}
      Because $u$ is smooth, $\Delta u$ is smooth, and in particular, $\Delta u$ is continuous. By Lebesgue Differentiation,
      \[ \Delta u(x) = \lim_{r \to 0^+} \fint_{B_r(x)}^{} \Delta u dx .\]

      By the Divergence Theorem, we know
      \[ \int_{B_r(x)}^{} \Delta u dx = \int_{\partial B_r(x)}^{} \frac{\partial u}{\partial \eta}(z) d \sigma(z) ,\]
      where $\eta$ is the unit outer normal to $B_r(x)$. Now we introduce the change of variables $z = x+ry$. Then we have
      \[ \int_{\partial B_r(x)}^{} \frac{\partial u}{\partial \eta}(z) d \sigma(z) = r^{n-1} \int_{B_1(0)}^{} \frac{\partial u}{\partial r} d
      \sigma(y) .\]
      The region we are integrating over no longer depends on $r$, so we can move the derivative outside to get
      \[ \int_{\partial B_r(x)}^{} \frac{\partial u}{\partial \eta}(z) d \sigma(z) = r^{n-1} \frac{\partial}{\partial r} \int_{\partial B_1(0)}^{}
      u d \sigma .\]

      Therefore,
      \begin{align*}
        \fint_{B_r(x)}^{} \Delta u dx
        &=\frac{1}{m(B_r)} \int_{B_r(x)}^{} \Delta u dx \\
        &= \frac{r^{n-1}}{m(B_r)} \frac{\partial}{\partial r} \int_{\partial B_1(0)}^{} u d\sigma \\
        &= \frac{1}{r} \frac{\partial}{\partial r} \frac{1}{m(B_1)} \int_{\partial B_1(0)}^{} u d\sigma \\
        &= \frac{n}{r} \frac{\partial}{\partial r} \frac{1}{m(\partial B_1)} \int_{\partial B_1(0)}^{} u d\sigma \\
        &= \frac{n}{r} \frac{\partial}{\partial r} \frac{1}{r^{n-1} m(\partial B_1)} \int_{\partial B_r(x)}^{} u d\sigma \quad \parbox{4cm}{after
        rescaling in the opposite way as before} \\
        &= \frac{n}{r} \frac{\partial}{\partial r} \frac{1}{m(\partial B_r)} \int_{\partial B_r(x)}^{} u d\sigma \\
        &= \frac{n}{r} \frac{\partial}{\partial r} \fint_{\partial B_r(x)}^{} u d\sigma
      \end{align*}

      \textbf{WORK BELOW HERE IS WRONG}
      Now we will define,
      \[ \varphi(r) = \fint_{\partial B_r(x)}^{} u d \sigma .\]
      Because $u$ is continuous, we can define
      \[ \varphi(0) = \lim_{r \to 0^+} \varphi(r) = u(x) \]
      using Lebesgue differentiation. Then
      \begin{align*}
        2 \frac{\varphi(r) - \varphi(0)}{r} \to \varphi'(r/2)
      \end{align*}
      Therefore,
      \begin{align*}
        \lim_{r \to 0^+} 2 \frac{\varphi(r) - \varphi(0)}{r} &= \lim_{r \to 0^+} \varphi'(r/2) \\
        &= \lim_{r \to 0^+} \varphi(r)
      \end{align*}

      This gives us
      \[ \lim_{r \to 0^+} \frac{n}{r} \frac{\partial}{\partial r} \fint_{\partial B_r(x)}^{} u d \sigma = \lim_{r \to 0^+} \frac{2n}{r^2} \left[ \fint_{\partial B_r(x)}^{} u
      d\sigma - u(x) \right] \]

      Combining this with the previous work, we have
      \begin{align*}
        \Delta u(x) &= \lim_{r \to 0^+} \fint_{B_r(x)}^{} \Delta u dx \\
        &= \lim_{r \to 0^+} \frac{n}{r} \frac{\partial}{\partial r} \fint_{\partial B_r(x)}^{} u d \sigma \\
        &= \lim_{r \to 0^+} \frac{2n}{r^2} \left[ \fint_{\partial B_r(x)}^{} u d\sigma - u(x) \right]
      \end{align*}

    \end{solution}

  \item Evans, Chapter 6 \#7 \\
    \begin{claim}
      Let $u \in H^1(\R^n)$ have compact support and be a weak solution of the semilinear PDE
      \[ - \Delta u + c(u) = f \quad \text{in } \R^n ,\]
      where $f \in L^2(\R^n)$ and $c: \R \to \R$ is smooth, with $c(0) = 0, c' \geq 0$, and $c(u) \in L^2$. Then $\|D^2 u \|_{L^2} \leq C \|f\|_L^2$.
    \end{claim}

    \begin{proof}
      Let $U \subset \R^n$ be compact such that $\supp{u} \subset U$. Because $u$ is a weak solution of
      \[ -\Delta u + c(u) = f ,\]
      we know
      \[ \sum_{i = 1}^n \int_{U}^{} u_{x_i} v_{x_i} dx = \int_{U}^{} (f - c(u)) v dx \]
      for all $v \in H_0^1(U)$.

      Let $|h| > 0$ be small, $k \in \{1,2,\dots,n\}$, and choose
      \[ v = -D_k^{-h} (D_k^h u) ,\]
      where $D_k^h$ is the difference quotient
      \[ D_k^h u(x) = \frac{u(x + h e_k) - u(x)}{h} .\]

      We will let
      \[ A = \sum_{i=1}^n \int_{U}^{} u_{x_i} v_{x_i} dx \]
      and
      \[ B = \int_{U}^{} (f - c(u)) v dx .\]
      First, we will look at $A$.
      We have integration by parts and product formulas for difference quotients given as
      \[ \int_{U}^{} v D_k^{-h} w dx = - \int_{U}^{} w D_k^h v dx ,\]
      and
      \[ D_k^h (vw) = v^h D_k^h w + w D_k^h v ,\]
      where $v^h(x) := v(x + he_k)$.
      So we have
      \begin{align*}
        A &= - \sum_{i=1}^n \int_{U}^{} u_{x_i} \left[ D_k^{-h} (D_k^h u) \right]_{x_i} dx \\
        &= \sum_{i=1}^n \int_{U}^{} D_k^h u_{x_i} \left( D_k^h u \right)_{x_i} dx \quad \\
        &= \sum_{i=1}^n \int_{U}^{} D_k^h u_{x_i} D_k^h u_{x_i} dx \\
        & = \| D (D_k^h u) \|_{L^2(U)}^2 \\
        &= \| D_k^h Du \|_{L^2(U)}^2
      \end{align*}

      Now we turn to $B$.
      By Theorem 3 in Section 5.8.2 of Evans,
      \begin{align*}
        \int_{U}^{} |v|^2 dx &= \int_{U}^{} |D_k^{-h} (D_k^h u)|^2 dx \\
        & \leq C \int_{U}^{} |D (D_k^h u)|^2 dx \\
        &= C \int_{U}^{} |D_k^h Du|^2 dx \quad \parbox{6cm}{NEED TO CHECK D AND $D_k^h$ COMMUTE}
      \end{align*}

      Therefore, by using Young's inequality we get that for any $\varepsilon > 0$,
      \begin{align*}
        |B| &= \int_{U}^{} (|f| + |c(u)|) |v| dx \\
        &\leq \int_{U}^{} \frac{|f|^2}{2 \varepsilon} + \frac{|c(u)|^2}{2 \varepsilon} + \varepsilon |v|^2 dx \\
        &\leq \int_{U}^{} \frac{|f|^2}{2 \varepsilon} + \frac{|c(u)|^2}{2 \varepsilon} + C\varepsilon |D_k^h Du|^2 dx
      \end{align*}

      Take $\varepsilon = \frac{1}{2C}$. Then combining $A$ and $B$, we have
      \begin{align*}
        \| D_k^h Du \|_{L^2(U)}^2 &= |A| \\
        &= |B| \\
        &\leq \int_{U}^{} C (|f|^2 + |c(u)|^2)dx + \frac{1}{2} \| D_k^h Du \|_{L^2(U)}^2
      \end{align*}
      Thus,
      \[ \| D_k^h Du \|_{L^2(U)}^2 \leq 2C \int_{U}^{} |f|^2 + |c(u)|^2 dx .\]

      Because we know $c(u) \in L^2$, we have
      \[ \|D_k^h Du \|_L^2(U)^2 \leq 2C \left( \|f\|_{L^2}^2 + \|c(u)\|_{L^2}^2 \right) .\]
      Because $c(u) \in L^2$, the right hand side is a finite constant, so by Theorem 3 in Section 5.8.2 of Evans, $Du \in H^1(U)$. Therefore, $u \in
      H^2(U)$, and $\|D^2 u\|_{L^2} \leq C' \|f\|_{L^2}$ for some constant $C'$.

    \end{proof}

  \item Evans, Chapter 6 \#8 \\
    \begin{claim}
      Let $u$ be a smooth solution of the uniformly elliptic equation $Lu = - \sum_{i,j=1}^n a^{ij}(x) u_{x_i x_j} = 0$ in $U$. Assume that the
      coefficients have bounded derivatives.
      Set $v := |Du|^2 + \lambda u^2$. Then
      \[ Lv \leq 0 \quad \text{in } U \]
      if $\lambda$ is large enough. Therefore,
      \[ \| Du \|_{L^\infty(U)} \leq C \left( \| Du \|_{L^\infty(\partial U)} + \| u \|_{L^\infty(\partial U)} \right). \]
    \end{claim}

    \begin{proof}
      We have $|Du|^2 = \sum_{k=1}^n u_{x_k}^2$. Therefore, we can compute
      \[ \left( |Du|^2 \right)_{x_i} = 2 \sum_{k=1}^n u_{x_k} u_{x_k x_i} \]
      and
      \[ \left( |Du|^2 \right)_{x_i x_j} = 2 \sum_{k=1}^n u_{x_k x_j} u_{x_k x_i} + 2 \sum_{k=1}^n u_{x_k} u_{x_k x_i x_j} .\]
      More simply, we have
      \[ (u^2)_{x_i} = 2 u u_{x_i} \]
      and
      \[ (u^2)_{x_i x_j} = 2 u_{x_j} u_{x_i} + 2 u u_{x_i x_j} .\]
      Plugging these in, we get
      \begin{align*}
        Lv &= - \sum_{i,j=1}^n a^{ij}(x) v_{x_i x_j} \\
        &= - 2 \sum_{i,j,k=1}^n a^{ij} u_{x_k x_j} u_{x_k x_i} - 2 \sum_{i,j,k=1}^n a^{ij} u_{x_k} u_{x_k x_i x_j} - 2 \lambda \sum_{i,j=1}^n a^{ij} u_{x_i} u_{x_j}
        - 2 \lambda \sum_{i,j=1}^n a^{ij} u u_{x_i x_j} \\
        &= - 2 \sum_{i,j,k=1}^n a^{ij} u_{x_k x_j} u_{x_k x_i} - 2 \sum_{i,j,i=1}^n a^{ij} u_{x_k} u_{x_k x_i x_j} - 2 \lambda \sum_{i,j=1}^n a^{ij} u_{x_i} u_{x_j}
      \end{align*}
      where we have used the fact that $Lu = 0$.

      We know for any $k$,
      \begin{align*}
        0 &= \frac{\partial}{\partial x_k} (Lu) \\
        &= \sum_{i,j=1}^n a^{ij}_{x_k} u_{x_i x_j} + \sum_{i,j=1}^n a^{ij} u_{x_i x_j x_k}
      \end{align*}
      Therefore, we can rewrite
      \[ Lv = -2 \sum_{i,j,k=1}^n a^{ij} u_{x_k x_j} u_{x_k x_i} + 2 \sum_{i,j,k=1}^n a^{ij}_{x_k} u_{x_k} u_{x_i x_j} - 2 \lambda \sum_{i,j=1}^n
      a^{ij} u_{x_i} u_{x_j} .\]

      For any fixed $k$, we have
      \begin{align*}
        \sum_{i,j=1}^n a^{ij} u_{x_k x_j} u_{x_k x_i} &= \sum_{i,j=1}^n a^{ij} (u_{x_k})_{x_j} (u_{x_k})_{x_i} \\
        &\geq \theta |D(u_{x_k})|^2 \quad \parbox{5cm}{by uniform ellipticity}
      \end{align*}
      for some $\theta > 0$.

      Also, by uniform ellipticity,
      \[ \sum_{i,j=1}^n a^{ij} u_{x_i} u_{x_j} \geq \theta |Du|^2 .\]

      By the uniform ellipticity and boundedness of derivatives of $a^{ij}$, we have
      \[ Lv \leq -2 \theta |D^2 u|^2 + A \sum_{i,j,k=1}^n u_{x_k} u_{x_i x_j} - 2 \theta \lambda |Du|^2 \]
      for some constant $A$. By Young's Inequality,
      \[ u_{x_k} u_{x_i x_j} \leq \frac{1}{2} u_{x_k}^2 + \frac{1}{2} u_{x_i x_j}^2 .\]
      Therefore, we get
      \begin{align*}
        Lv &\leq -2 \theta |D^2u|^2 + \frac{A n^2}{2} |Du|^2 + \frac{A n}{2} |D^2 u|^2 - 2\theta \lambda |Du|^2 \\
        &= \left( \frac{A n}{2} - 2 \theta \right) |D^2 u|^2 + \left(\frac{A n^2}{2} - 2 \theta \lambda \right) |Du|^2
      \end{align*}
      From here, we can pick $\lambda$ large enough to guarantee $Lv \leq 0$.

      By the maximum principle, we know
      \[ \| u \|_{L^\infty(U)} \leq \| u \|_{L^\infty(\partial U)} \]
      and
      \[ \left\| |Du|^2 + \lambda u^2 \right\|_{L^\infty(U)} = \| v \|_{L^\infty(U)} \leq \| v \|_{L^\infty(\partial U)} = \left\| |Du|^2 + \lambda u^2
      \right\|_{L^\infty(\partial U)} .\]

      By the triangle inequality,
      \[ \left\| |Du|^2 \right\|_{L^\infty(U)} \leq \left\| |Du|^2 + \lambda u^2 \right\|_{L^\infty(U)} + \lambda \| u^2 \|_{L^\infty(U)} .\]
      Therefore,
      \begin{align*}
        \left\| |Du|^2 \right\|_{L^\infty(U)} - \lambda \|u^2\|_{L^\infty(U)}
        &\leq \left\| |Du|^2 + \lambda u^2 \right\|_{L^\infty(U)} \\
        &\leq \left\| |Du|^2 + \lambda u^2 \right\|_{L^\infty(\partial U)} \\
        &\leq \left\| |Du|^2 \right\|_{L^\infty(\partial U)} + \lambda \|u^2\|_{L^\infty(\partial U)}
      \end{align*}

      Adding $\lambda \|u^2\|_{L^\infty(U)}$ to both sides and using the maximum principle gives
      \begin{align*}
        \left\| |Du|^2 \right\|_{L^\infty(U)} &\leq \left\| |Du|^2 \right\|_{L^\infty(\partial U)} + 2\lambda \|u^2\|_{L^\infty(\partial U)} \\
        &\leq \left( \left\| Du \right\|_{L^\infty(\partial U)} + \sqrt{2 \lambda} \|u\|_{L^\infty(\partial U)} \right)^2
      \end{align*}

      Taking square roots gives
      \[ \| Du \|_{L^\infty(U)} \leq C \left( \| Du \|_{L^\infty(\partial U)} + \|u\|_{L^\infty(\partial U)} \right) \]
      with $C = \max \{1, \sqrt{2 \lambda} \}$.

    \end{proof}

  \item Evans, Chapter 6 \#10 \\
    \begin{problem}
      Assume $U$ is connected. Use (a) energy methods and (b) the maximum principle to show that the only smooth solutions of the Neumann
      boundary-value problem
      \[ \begin{cases}
          - \Delta u = 0 &\text{in } U \\
          \frac{\partial u}{\partial \nu} = 0 &\text{on } \partial U
      \end{cases} \]
      are $u \equiv C$, for some constant $C$.
    \end{problem}

      \begin{proof}
        \begin{enumerate}
          \item
            Let $u$ satisfy the given Neumann problem. Multiplying $-\Delta u$ by $u$ and integrating gives
            \begin{align*}
              0 &= \int_{U}^{} - \Delta u u dx \\
              &= \int_{U}^{} \nabla u \cdot \nabla u dx - \int_{\partial U}^{} u \nabla u \cdot \nu dS \\
              &= \int_{U}^{} \nabla u \cdot \nabla u dx \quad \parbox{5cm}{because $\frac{\partial u}{\partial \nu} =0$ on $\partial U$} \\
              &= \| \nabla u \|_{L^2(U)}^2
            \end{align*}
            Therefore, $\nabla u (x) = 0$ for every $x \in U$. That is, $u \equiv C$ for some constant $C$.

          \item
            Assume $u$ is not constant. By the maximum principle, there is an $x_0 \in \partial U$ such that $u(x_0) > u(x)$ for all $x \in U$ ($U$ is
            open, so this is saying $u$ attains its maximum on the boundary but not within the interior).

            Assuming we have the interior ball condition, we can take a ball $B$ such that $B \subset U$ with $x_0 \in \partial B$. By our use of the
            maximum principle, we know $u(x_0) > u(x)$ for all $x \in U$, so we can apply Hopf's Lemma to find that
            \[ \frac{\partial u}{\partial \tilde{\nu}} (x_0) > 0 \]
            where $\tilde{\nu}$ is the outer unit normal to $B$.

            \textbf{MUST SHOW UNIT NORMAL TO BALL IS SAME AS UNIT NORMAL TO $U$}

            Therefore,
            \[ \frac{\partial u}{\partial \nu} > 0 \]
            where $\nu$ is the outer unit normal to $U$. This is a contradiction, so $u$ must be constant.
        \end{enumerate}
      \end{proof}

  \item Renardy-Rogers, pg. 225 \#5 \\
    \begin{claim}
      Assume $u,v \in H^1(\R)$. Then $uv \in H^1(\R)$.
    \end{claim}

    \begin{proof}
      Take $u,v \in H^1(\R)$. By Theorem 7.18 in Renardy-Rogers (Sobolev Embedding), there is a constant C with
      \[ \|u\|_{L^\infty} \leq C \|u\|_{H^1} \]
      and
      \[ \|v\|_{L^\infty} \leq C \|v\|_{H^1} .\]
      Then we have
      \begin{align*}
        \|uv\|_{H^1}^2 &= \int_{\R}^{} |uv|^2 + |(uv)'|^2 dx \\
        &\leq \int_{\R}^{} |uv|^2 + |u' v|^2 + |u v'|^2 dx \\
        &= \int_{\R}^{} u^2 (v^2 + (v')^2) + (u')^2 v^2 dx \\
        &\leq \int_{\R}^{} u^2 (v^2 + (v')^2) + v^2 (u^2 + (u')^2) dx \\
        &\leq \|u\|_{L^\infty}^2 \int_{\R}^{} v^2 + (v')^2 dx + \|v\|_{L^\infty}^2 \int_{\R}^{} u^2 + (u')^2 dx \\
        &\leq C \|u\|_{H^1}^2 \|v\|_{H^1}^2 + C \|u\|_{H^1}^2 \|v\|_{H^1}^2 \\
        &< \infty
      \end{align*}

      Therefore $uv \in H^1(\R)$.
    \end{proof}

  \item Renardy-Rogers, pg. 225 \#8 \\
    \begin{claim}
      Let $\Omega$ satisfy the assumptions of Theorem 7.29. Then every weakly convergent sequence in $H^{k+1}(\Omega)$ converges strongly in
      $H^k(\Omega)$.
    \end{claim}

    \begin{proof}
      First, we need to know that a weakly convergent sequence is bounded.
      \begin{lemma} \label{weak_bdd}
        Let $H$ be a Hilbert space. Assume $u_n \rightharpoonup u$ in $H$. Then $\{u_n\}$ is a bounded sequence.
      \end{lemma}
      \begin{proof}
        Define the maps $T_n:H \to \R$ by
        \[ T_n(v) = \la u_n, v \ra .\]
        For any fixed $v \in H$,
        \[ T_n(v) = \la u_n, v \ra \to \la u, v \ra \]
        by the weak convergence. Therefore, $\{ T_n(v) \}_{n=1}^\infty$ is a bounded set of real numbers.
        By the Uniform Boundedness Principle,
        \[ \sup_n \|T_n\| < \infty \]
        where the norm on $T_n$ is the operator norm.
        But by the definition of the operator norm,
        \[ \|T_n\| = \sup_{\|v\|=1} |T_n(v)| = \sup_{\|v\|=1} |\la u_n, v \ra| \leq \|u_n\| .\]
        Also,
        \[ \|T_n\| \geq \frac{T_n(u_n)}{\|u_n\|} = \\|u_n\| .\]
        Thus, $\|T_n\| = \|u_n\|$. By our use of the Uniform Boundedness Principle, we then have
        \[ \sup_n \|u_n\| = \sup_n\|T_n\| < \infty .\]
        That is, $\{u_n\}$ is a bounded sequence.
      \end{proof}

      Let $u_n \rightharpoonup u$ in $H^{k+1}(\Omega)$. By Lemma \ref{weak_bdd}, $\{u_n\}$ is a bounded sequence.
      By Theorem 7.29 in Renardy-Rogers, $H^{k+1}(\Omega)$ is compactly embedded in $H^k(\Omega)$. By definition, this means $\{u_n\}$ contains a
      subsequence $\{u_{n_k}\}$ such that
      \[ u_{n_k} \to u \quad \text{in } H^{k}(\Omega). \]
    \end{proof}

  \item Renardy-Rogers, pg. 225 \#10 \\
    \begin{claim}
      Let $\Omega$ be bounded (no assumptions on $\partial \Omega$). Then $H_0^{k+1}(\Omega)$ is compactly embedded in $H^k(\Omega)$.
    \end{claim}

    \begin{proof}
      Let $\{u_n\}$ be a bounded sequence in $H^{k+1}_0(\Omega)$. Let $M_R$ denote the operator of multiplication by $\chi_{ \{|\xi| < R \} }$. For $ v
      \in H^{k+1}(\R^m)$, let ${v^R = \mathcal{F}^{-1} \left[ M_R \mathcal{F} (v) \right] }$.
      Then
      \begin{align*}
        \| u - u^R \|_{H^k}^2 &= \left\| (1 + |\xi|^{2k})^{1/2} \mathcal{F} (u - u^R) \right\|_{L^2}^2 \\
        &= \left\| (1 + |\xi|^{2k})^{1/2} ( \mathcal{F}u - M_R \mathcal{F} u ) \right\|_{L^2}^2 \\
        &= \int_{-\infty}^{\infty} (1 + |\xi|^{2k}) ( \mathcal{F} u - M_R \mathcal{F} u)^2 d\xi \\
        &= \int_{|\xi|>R}^{} (1 + |\xi|^{2k}) |\hat{u} (\xi)|^2 d\xi \\
        &= \int_{|\xi|>R}^{} (1 + |\xi|^{2k}) \left| \frac{2 \pi i \xi}{2 \pi i \xi} \hat{u}(\xi) \right|^2 d\xi \\
        &\leq \frac{1}{(2 \pi R)^2} \int_{|\xi|>R}^{} (1 + |\xi|^{2k}) | \widehat{u'} (\xi) |^2 d \xi \\
        &= \frac{1}{(2 \pi R)^2} \|u'\|_{H^k}^2 \\
        &\leq \frac{1}{(2 \pi R)^2} \|u\|_{H^{k+1}}^2
      \end{align*}

      Therefore,
      \begin{equation} \label{cutoff_bound}
        \|u - u^R\|_{H^k} \leq \frac{C}{R} \|u\|_{H^{k+1}} ,
      \end{equation}
      for some constant $C$.

      Moreover, for any fixed $R$ and any positive integer $l$, we have $u^R \in C^l(\R^m)$ and there is a constant $C(R,l)$ such that
      \[ \|u^R\|_{W^{l,\infty}} \leq C(R,l) \|u\|_{H^{k+1}} .\]
      Therefore, for any fixed $R, \{u^R_n\}_n$ is a uniformly bounded sequence which is equicontinuous (derivatives are bounded).
      By Arzela-Ascoli, $\{u^R_n\}_n$ has a subsequence that converges in $C^k(\Omega)$ and therefore in $H^k(\Omega)$. For convenience, we shall
      denote the subsequence as $\{u_n^R\}_n$ as well.
      Now, choose a sequence of radii $\{R_k\}_k$ with $R_k \to \infty$.
      By choosing the sequence $\{u_n^{R_n} \}_n$, we get a sequence that converges in $H^k(\Omega)$.

      By \eqref{cutoff_bound}, we have
      \[ \| u_n - u_n^{R_n} \|_{H^k} \leq \frac{C}{R_n} \|u_n\|_{H^{k+1}} .\]
      The right hand side above approaches 0 as $n \to \infty$ because $\{u_n\}$ is a bounded sequence in $H^{k+1}_0(\Omega)$. Therefore, $u_n$
      converges in $H^k(\Omega)$.
    \end{proof}

\end{enumerate}
\end{document}



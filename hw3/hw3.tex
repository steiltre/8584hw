%        File: hw3.tex
%     Created: Sun Apr 10 07:00 PM 2016 C
% Last Change: Sun Apr 10 07:00 PM 2016 C
%

\documentclass[a4paper]{article}

\title{Math 8584 Homework 3 }
\date{4/18/16}
\author{Trevor Steil}

\usepackage{amsmath}
\usepackage{amsthm}
\usepackage{amssymb}

\newtheorem{theorem}{Theorem}[section]
\newtheorem{corollary}{Corollary}[section]
\newtheorem{proposition}{Proposition}[section]
\newtheorem{lemma}{Lemma}[section]
\newtheorem*{claim}{Claim}
\newtheorem*{problem}{Problem}
%\newtheorem*{lemma}{Lemma}
\newtheorem{definition}{Definition}[section]

\newcommand{\R}{\mathbb{R}}
\newcommand{\N}{\mathbb{N}}
\newcommand{\C}{\mathbb{C}}
\newcommand{\Z}{\mathbb{Z}}
\newcommand{\supp}[1]{\mathop{\mathrm{supp}}\left(#1\right)}
\newcommand{\lip}[1]{\mathop{\mathrm{Lip}}\left(#1\right)}
\newcommand{\curl}{\mathrm{curl}}
\newcommand{\la}{\left \langle}
\newcommand{\ra}{\right \rangle}
\renewcommand{\vec}[1]{\mathbf{#1}}

\newenvironment{solution}{\emph{Solution.}}

\begin{document}
\maketitle
\begin{enumerate}
  \item Jost, pg 91 Formula 5.1.15
    \begin{claim}
      Let $u: \R^d \times \R \to \R$ satisfy
      \begin{equation*}
        u_t = \Delta u.
      \end{equation*}
      Then
      \[ u(y,T) = \mu \int_{\partial M(y, T; \mu)}^{} u(x,t) \frac{|x-y|}{2(T-t)} d \sigma \]
      where
      \[ M(y,T; \mu) = \left\{ (x,s) \in \R^d \times \R, s \leq T : \frac{1}{(4 \pi (T-s))^{d/2}} e^{-\frac{|x-y|^2}{4(T-s)} } \geq \mu \right\} .\]
    \end{claim}

    \begin{proof}
      We define
      \[ \Lambda(x,y,t,t_0) = \frac{1}{(4 \pi |t-t_0|)^{d/2}} e^{\frac{|x-y|^2}{4(t_0 - t)} } .\]
      Because $u$ satisfies the heat equation, we have
      \begin{align*}
        0 &= \iint_{M(y,T;\mu)}^{} ( \Lambda(x,y,T+\varepsilon,t) - \mu) (u_t (x,t) - \Delta u(x,t)) dx dt \\
        &= \iint_{M(y,T;\mu)}^{} ( \Lambda(x,y,T+\varepsilon,t) - \mu ) u_t(x,t) dt dx
        - \iint_{M(y,T,\mu)}^{} ( \Lambda(x,y,T+\varepsilon,t) - \mu ) \Delta u(x,t) dx dt \\
        &=
      \end{align*}<++>
    \end{proof}

  \item Jost 5.1
    \begin{claim}
      Let $\Omega \subset \R^d$ be bounded, $\Omega_T = \Omega \times (0,T)$. Let
      \[ L = \sum_{i,j=1}^d a^{ij}(x,t) \frac{\partial^2}{\partial x^i \partial x^j} + \sum_{i=1}^d b^i(x,t) \frac{\partial}{\partial x^i} .\]
      be elliptic for all $(x,t) \in \Omega_T$, and suppose
      \[ u_t \leq Lu, \]
      where $u \in C^0(\overline{\Omega_T})$ is twice continuously differentiable with respect to $x \in \Omega$ and once with respect to $t \in
      (0,T)$. Then
      \[ \sup_{\Omega_T} u = \sup_{\partial^\ast \Omega_T} u. \]
    \end{claim}

    \begin{proof}
      Assume $T < \infty$ and $u_t - Lu < 0$ in $\Omega_T$.
      Fix $0<\varepsilon<T$. Suppose we have $(x_0,t_0) \in \overline{\Omega}_{T-\varepsilon}$ such that
      \[ \max_{\overline{\Omega}_{T-\varepsilon}} u = u(x_0,t_0) .\]

    \end{proof}

  \item Jost 5.4
    \begin{claim}
      Let $\Sigma$ be the grid consisting of the points $(x,t)$ with $x = nh, t = mk, n,m \in \Z, m \geq 0$, and let $v$ be the solution of the
      discrete heat equation
      \[ \frac{v(x,t+k) - v(x,t)}{k} - \frac{v(x+h,t) - 2v(x,t) + v(x-h,t)}{h^2} = 0 \]
      with $v(x,0) = f(x) \in C^0(\R)$.
      Then for $\frac{k}{h^2} = \frac{1}{2}$,
      \[ v(nh, mk) = 2^{-m} \sum_{j=0}^m \binom{m}{j} f \left( (n - m + 2j)h \right) .\]
      Also,
      \[ \sup_{\Sigma} |v| \leq \sup_\R |f|. \]
    \end{claim}

    \begin{proof}
    \end{proof}

  \item Evans pg 87 \#12
    \begin{claim}
      Suppose $u$ is smooth and solves $u_t - \Delta u = 0$ in $\R^n \times (0,\infty)$. Then
      \begin{enumerate}
        \item $u_\lambda(x,t) = u(\lambda x, \lambda^2 t)$ also solves the heat equation for each $\lambda \in \R$.

        \item $v(x,t) = x \cdot Du(x,t) + 2t u_t(x,t)$ solves the heat equation as well.
      \end{enumerate}

    \end{claim}

    \begin{proof}
      \begin{enumerate}
        \item
          A simple calculation shows
          \begin{align*}
            \partial_t u_\lambda + \Delta u_\lambda &= \lambda^2 u_t + \lambda^2 \Delta u \\
            &= \lambda^2 (u_t + \Delta u) \\
            &= 0
          \end{align*}

        \item
          By part (a), we know $u_\lambda(x,t) = u(\lambda x, \lambda^2 t)$ satisfies the heat equation.
          Write
          \[ u_\lambda(x, t) = u( \lambda x_1, \dots, \lambda x_n, \lambda^2 t) .\]
          Taking the derivative with respect to $\lambda$ gives
          \begin{align*}
            \partial_\lambda u_\lambda &= x_1 \partial_{x_1} u + \dots + \partial_{x_n} u + 2 \lambda t \partial_t u \\
            &= x \cdot Du + 2 \lambda t u_t
          \end{align*}
          By choosing $\lambda = 1$, we get $v$.
          By assumption, $u$ is smooth, so our partial derivatives commute. Therefore,
          \begin{align*}
            v_t - \Delta v &= \partial_t ( \partial_\lambda u_\lambda |_{\lambda=1} ) - \Delta ( \partial_\lambda u_\lambda |_{\lambda=1} ) \\
            &= \partial_\lambda (\partial_t u_\lambda ) |_{\lambda=1} + \partial_\lambda (\Delta u_\lambda) |_{\lambda=1} \\
            &= \partial_\lambda (\partial_t u_\lambda + \Delta u_\lambda ) |_{\lambda=1} \\
            &= 0
          \end{align*}
          because $u_\lambda$ solves the heat equation. Thus, $v$ solves the heat equation as well.
      \end{enumerate}
    \end{proof}

  \item Evans pg 87 \#13
    \begin{problem}
      Assume $n = 1$ and $u(x,t) = v\left( \frac{x}{\sqrt{t}} \right)$.
      \begin{enumerate}
        \item Show
          \[ u_t = u_{xx} \]
          if and only if
          \begin{equation} \label{eq:ODE}
            v'' + \frac{z}{2} v' = 0
          \end{equation}

          Also, show that the general solution of \eqref{eq:ODE} is
          \[ v(z) = c \int_{0}^{z} e^{-s^2/4} ds + d .\]

        \item
          Differentiate $u(x,t) = v\left( \frac{x}{\sqrt{t}} \right)$ with respect to $x$ and select the constant $c$ properly, to obtain the
          fundamental solution $\Phi$ for $n = 1$. Explain why this procedure produces the fundamental solution.

      \end{enumerate}

    \end{problem}

    \begin{solution}
      \begin{enumerate}
        \item

        \item
      \end{enumerate}
    \end{solution}
\end{enumerate}
\end{document}



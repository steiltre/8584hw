%        File: final.tex
%     Created: Sat May 07 10:00 PM 2016 C
% Last Change: Sat May 07 10:00 PM 2016 C
%

\documentclass[a4paper]{article}

\title{Math 8584 Final }
\date{5/12/16}
\author{Trevor Steil}

\usepackage{amsmath}
\usepackage{amsthm}
\usepackage{amssymb}
\usepackage{esint}

\newtheorem{theorem}{Theorem}[section]
\newtheorem{corollary}{Corollary}[section]
\newtheorem{proposition}{Proposition}[section]
\newtheorem{lemma}{Lemma}[section]
\newtheorem*{claim}{Claim}
\newtheorem*{problem}{Problem}
%\newtheorem*{lemma}{Lemma}
\newtheorem{definition}{Definition}[section]

\newcommand{\R}{\mathbb{R}}
\newcommand{\N}{\mathbb{N}}
\newcommand{\C}{\mathbb{C}}
\newcommand{\Z}{\mathbb{Z}}
\newcommand{\supp}[1]{\mathop{\mathrm{supp}}\left(#1\right)}
\newcommand{\lip}[1]{\mathop{\mathrm{Lip}}\left(#1\right)}
\newcommand{\curl}{\mathrm{curl}}
\newcommand{\la}{\left \langle}
\newcommand{\ra}{\right \rangle}
\renewcommand{\vec}[1]{\mathbf{#1}}
\renewcommand{\div}{\mathrm{div}}

\newenvironment{solution}{\emph{Solution.}}

\begin{document}
\maketitle

\begin{enumerate}
  \item McOwen Page 398, Problem 1
    \begin{claim} $G(u) = \int_{}^{} a(x) |u|^{\sigma + 1} dx$, as in Proposition 1, is a $C^1$ functional.
    \end{claim}

    \begin{proof}
      The book already shows $G(u)$ is differentiable. We must only show $G'$ is continuous.
    \end{proof}

    \pagebreak

  \item McOwen Page 398, Problem 2
    \begin{problem}
      This exercise shows that the restriction $1 < \sigma < (n+2)/(n-2)$ is necessary for the existence of positive solutions of
      \[ \begin{cases}
          \Delta u + a(x) u^\sigma = 0 &\text{in } \Omega \\
          u = 0 &\text{on } \partial \Omega
        \end{cases} \]
        by considering
        \begin{equation}
          \begin{cases}
            \Delta u + f(u) = 0 &\text{in } \Omega \\
            u=0 &\text{on } \partial \Omega
          \end{cases}
          \label{eqn:prob2}
        \end{equation}
        where $f(u)$ is continuous in $u \in \R$.

        \begin{enumerate}
          \item
            If $F(u) = \int_{0}^{u} f(t)dt$, use integration by parts to show
            \[n \int_{\Omega}^{} F(u) dx + \int_{\Omega}^{} f(u) \sum_{i=1}^n x_i \frac{\partial u}{\partial x_i} dx = 0 ,\]
            for any $u \in C^1( \overline{\Omega} )$ with $u=0$ on $\partial \Omega$.

          \item
            If $u \in C^2(\Omega) \cap C( \overline{\Omega} )$ satisfies \eqref{eqn:prob2}, then prove Pohozaev's identity
            \[ \frac{n-2}{2} \int_{\Omega}^{} |\nabla u|^2 dx - n \int_{\Omega}^{} F(u)dx + \frac{1}{2} \int_{\partial \Omega}^{} \left(
            \frac{\partial u}{\partial \nu} \right)^2 (x \cdot \nu) ds = 0 \]
            where $\nu$ is the exterior unit normal.

          \item
            When $\Omega$ is a ball in $\R^n$, show that \eqref{eqn:prob2} admits no positive solutions $u \in C^2 ( \Omega ) \cap C(
            \overline{\Omega} )$ when $f(u) = u^\sigma$ with $\sigma \geq (n+2)/(n-2)$.

        \end{enumerate}

      \end{problem}

      \begin{solution}

        \begin{enumerate}
          \item
            By writing $n F(u) = \sum_{i=1}^n F(u) \partial_i x_i$, we get
            \begin{align*}
              n \int_{\Omega}^{} F(u) dx &= \int_{\Omega}^{} \sum_{i=1}^n F(u) \partial_i x_i dx \\
              &= \int_{\partial \Omega}^{} \sum_{i=1}^n F(u) x_i \nu_i ds - \int_{\Omega}^{} \sum_{i=1}^n \partial_i F(u) x_i dx \quad
              \parbox{5cm}{integrating by parts} \\
              &= - \int_{\Omega}^{} \sum_{i=1}^n \partial_i \int_{0}^{u} f(t) dt x_i dx \quad \parbox{5cm}{because $u=0$ on $\partial \Omega$}\\
              &= - \int_{\Omega}^{} \sum_{i=1}^n f(u) \frac{\partial u}{\partial x_i} x_i dx
            \end{align*}

            Therefore,
            \[ n \int_{\Omega}^{} F(u) dx + \int_{\Omega}^{} f(u) \sum_{i=1}^n x_i \frac{\partial u}{\partial x_i} dx = 0 .\]

          \item
            Using part (a) and Einstein summation notation, we have
            \begin{align*}
              n \int_{\Omega}^{} F(u) dx &= - \int_{\Omega}^{} f(u) \partial_i u x_i dx \\
              &= \int_{\Omega}^{} \Delta u \partial_i u x_i dx \quad \parbox{5cm}{by \eqref{eqn:prob2}} \\
              &= \int_{\Omega}^{} \partial_j \partial_j u \partial_i u x_i dx \\
              &= \int_{\partial \Omega}^{} x_i \partial_i u \partial_j u \nu_j ds - \int_{\Omega}^{} \partial_j u \partial_j (x_i \partial_i u) dx \\
              &= \int_{\partial \Omega}^{} x_i \partial_i u \partial_j u \nu_j ds - \int_{\Omega}^{} \partial_j u (\delta_{ij} \partial_i u + x_i
              \partial_j \partial_i u ) dx \\
              &= \int_{\partial \Omega}^{} x_i \partial_i u \partial_j u \nu_j ds - \int_{\Omega}^{} \partial_j u \partial_j u dx -
              \int_{\Omega}^{} x_i \partial_j u \partial_j \partial_i u dx \\
              &= \int_{\partial \Omega}^{} x_i \partial_i u \partial_j u \nu_j ds - \int_{\Omega}^{} | \nabla u |^2 dx - \frac{1}{2}
              \int_{\Omega}^{} x_i \partial_i (\partial_j u \partial_j u) dx \\
              &= \int_{\partial \Omega}^{} x_i \partial_i u \partial_j u \nu_j ds - \int_{\Omega}^{} | \nabla u |^2 dx - - \frac{1}{2} \int_{\partial
              \Omega}^{} x_i \nu_i | \nabla u |^2 ds + \frac{n}{2} \int_{\Omega}^{} | \nabla u |^2 dx \\
              &= \frac{n-2}{2} \int_{\Omega}^{} | \nabla u |^2 dx + \int_{\partial \Omega}^{} x_i \partial_i u \partial_j u \nu_j ds -
              \frac{1}{2} \int_{\Omega}^{} x_i \nu_i | \nabla u |^2 ds \\
              &=
            \end{align*}

            Because $u=0$ on $\partial \Omega$, $\partial \Omega$ is a level set for $u$, and $| \nabla u |^2 = \left( \frac{\partial u}{\partial \nu}
            \right)^2$.

          \item

        \end{enumerate}

      \end{solution}

      \pagebreak

  \item McOwen, Page 237, Problem 7
    \begin{problem}
      Prove the following relationship between the Dirichlet eigenvalues $\lambda_n$ and Neumann eigenvalues $\mu_n$ of the Laplacian on a bounded
      domain with $C^1$-boundary: $\lambda_n \geq \mu_n$ for all $n \geq 1$.
    \end{problem}

    \begin{solution}
    \end{solution}

    \pagebreak

  \item McOwen Page 227, Problem 8
    \begin{problem}
      Define the graph area functional by
      \[ F(u) = \int_{\Omega}^{} \sqrt{1 + u_x^2 + u_y^2} dx dy \quad \text{for} u \in H^1(\Omega) \]
      where $\Omega$ is a bounded domain in $\R^2$.

      \begin{enumerate}
        \item Compute the derivative of $F$ and show that $F$ is $C^1$ on $H^1(\Omega)$.

        \item
          Given $g \in H^1(\Omega)$, let
          \[ \mathcal{A} = \{ u \in H^1(\Omega) : u-g \in H^1_0(\Omega) \} = \{u = g+v : v \in H^1_0(\Omega) \} .\]
          Show that a critical point of $F$ on $\mathcal{A}$ is a weak solution of the minimal surface equation $(1+u_y^2)u_{xx} - 2 u_x u_y u_{xy} +
          (1+u_x^2)u_{yy} = 0$ in $\Omega, u=g$ on $\partial \Omega$.

        \item
          Generalize (a) and (b) to dimension $n$, obtaining the minimal surface equation given by
          \[ \div \left( \frac{\nabla u}{(1+| \nabla u |^2)^{1/2}} \right) = 0 .\]

      \end{enumerate}

    \end{problem}

    \begin{solution}
    \end{solution}

    \pagebreak

  \item McOwen Page 228, Problem 11
    \begin{problem}
      Suppose $F$ is a $C^1$ functional on $X$ satisfying (i) $F'(u) = L + K(u)$, where $L:X \to X^\ast$ is an isomorphism (i.e., a bounded linear
      operator that is one-to-one and onto) and $K:X \to X^\ast$ is a compact map (i.e., maps bounded sets of $X$ to precompact sets of $X^\ast$ but
      $K$ is not necessarily linear); and (ii) every Palais-Smale sequence is bounded in $X$. Show that $F$ satisfies the Palais-Smale condition.
    \end{problem}

    \begin{solution}
    \end{solution}

\end{enumerate}
\end{document}



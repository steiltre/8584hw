%        File: final.tex
%     Created: Sat May 07 10:00 PM 2016 C
% Last Change: Sat May 07 10:00 PM 2016 C
%

\documentclass[a4paper]{article}

\title{Math 8584 Final }
\date{5/12/16}
\author{Trevor Steil}

\usepackage{amsmath}
\usepackage{amsthm}
\usepackage{amssymb}
\usepackage{esint}

\newtheorem{theorem}{Theorem}[section]
\newtheorem{corollary}{Corollary}[section]
\newtheorem{proposition}{Proposition}[section]
\newtheorem{lemma}{Lemma}[section]
\newtheorem*{claim}{Claim}
\newtheorem*{problem}{Problem}
%\newtheorem*{lemma}{Lemma}
\newtheorem{definition}{Definition}[section]

\newcommand{\R}{\mathbb{R}}
\newcommand{\N}{\mathbb{N}}
\newcommand{\C}{\mathbb{C}}
\newcommand{\Z}{\mathbb{Z}}
\newcommand{\supp}[1]{\mathop{\mathrm{supp}}\left(#1\right)}
\newcommand{\lip}[1]{\mathop{\mathrm{Lip}}\left(#1\right)}
\newcommand{\curl}{\mathrm{curl}}
\newcommand{\la}{\left \langle}
\newcommand{\ra}{\right \rangle}
\renewcommand{\vec}[1]{\mathbf{#1}}
\renewcommand{\div}{\mathrm{div}}

\newenvironment{solution}{\emph{Solution.}}

\begin{document}
\maketitle

\begin{enumerate}
  \item McOwen Page 398, Problem 1
    \begin{claim} $G(u) = \int_{}^{} a(x) |u|^{\sigma + 1} dx$, as in Proposition 1, is a $C^1$ functional.
    \end{claim}

    \begin{proof}
      The book already shows $G(u)$ is differentiable. We must only show $G'$ is continuous.

      We define the function
      \[ f(t) = |t|^{\sigma-1}t .\]
      We find $f'(t) = \sigma |t|^{\sigma-1}$ and $f''(t) = \sigma (\sigma -1) t |t|^{\sigma-3}$. By Taylor's Theorem, we know
      \[ f(b) = f(a) + f'(a)(b-a) + \frac{f''(c)}{2} (b-a)^2 \]
      for some $c \in (a,b)$. Substituting $b = u+w$ and $a = u$, we have
      \[ |u+w|^{\sigma-1} (u+w) = |u|^{\sigma-1} u + \sigma |u+\theta w|^{\sigma-1}w \]
    for some $\theta \in (0,1)$.

    Using this result, we get for $u,v,w \in H^1_0(\Omega)$
    \begin{align*}
      |(G'(u+w) - G'(u))v| &= (\sigma+1) \left| \int_{\Omega}^{} a(x) \left( |u+w|^{\sigma-1}(u+w) - |u|^{\sigma-1}u \right) v dx \right| \\
      &= (\sigma + 1) \left| \int_{\Omega}^{} a(x) \sigma |u + \theta w|^{\sigma-1} w v dx \right|
    \end{align*}

    Because $(a+b)^r \leq 2^r(a^r + b^r)$ for $a,b,r>0$, we have
    \begin{align*}
      |u+\theta w|^{\sigma-1} &\leq ( |u| + \theta |w| )^{\sigma-1} \\
      &\leq C ( |u|^{\sigma-1} + |w|^{\sigma-1} )
    \end{align*}
    for some $C>0$. Therefore, we have
    \begin{equation*}
      |(G'(u+w) - G'(u))v| \leq C \left| \int_{\Omega}^{} |u|^{\sigma-1} w v dx + \int_{\Omega}^{} |w|^{\sigma-1}w v dx \right|
    \end{equation*}
    where we have used the fact that $a$ is smooth and $\Omega$ is bounded to get a bound on $a$.

    By H\"{o}lder's inequality,
    \[ \left|\int_{\Omega}^{} |w|^{\sigma-1} w v dx \right| \leq \left( \int_{\Omega}^{} |w|^{\sigma p} dx \right)^{\frac{1}{p}} \left( \int_{\Omega}^{} |v|^{p'} dx
    \right)^{\frac{1}{p'}} .\]
    where $\frac{1}{p} + \frac{1}{p'} = 1$. By choosing $p=\frac{2n}{n+2}$ and $p'=\frac{2n}{n-2}$ we are able to use Sobolev Embedding to get
    \[ \left| \int_{\Omega}^{} |w|^{\sigma-1} w v dx \right| \leq \|w\|_{H^1_0}^\sigma \|v\|_{H^1_0} .\]

    By again using H\"{o}lder, we have
    \begin{equation*}
      \left| \int_{\Omega}^{} |u|^{\sigma-1} wv dx \right| \leq \left( \int_{\Omega}^{} |u|^{(\sigma-1)p} dx \right)^{\frac{1}{p}} \left(
      \int_{\Omega}^{} |w|^{p'} dx \right)^{\frac{1}{p'}} \left( \int_{\Omega}^{} |v|^{p'} dx \right)^{\frac{1}{p'}}
    \end{equation*}
    where $\frac{1}{p} + \frac{2}{p'} = 1$ By choosing $p=\frac{n}{2}$ and $p'=\frac{2n}{n-2}$, we can again use Sobolev Embedding to get
    \begin{equation*}
      \left| \int_{\Omega}^{} |u|^{\sigma-1} w v dx \right| \leq \|u\|_{H^1_0}^{\sigma-1} \|w\|_{H^1_0} \|v\|_{H^1_0}
    \end{equation*}

    Therefore,
    \begin{equation*}
      |(G'(u+w) - G'(u))v| \leq C(\|u\|_{H^1_0}^{\sigma-1} - 1) \|w\|_{H^1_0} \|v\|_{H^1_0}
    \end{equation*}

    By dividing by $\|v\|_{H^1_0}$ and using the fact that $u \in H^1_0(\Omega)$, we see
    \begin{align*}
      \frac{| (G'(u+w) - G'(u))v |}{\|v\|_{H^1_0}} &\leq C \|w\|_{H^1_0}
    \end{align*}

    Thus, $G'$ is continuous, and $G$ is $C^1$.

    \end{proof}

    \pagebreak

  \item McOwen Page 398, Problem 2
    \begin{problem}
      This exercise shows that the restriction $1 < \sigma < (n+2)/(n-2)$ is necessary for the existence of positive solutions of
      \[ \begin{cases}
          \Delta u + a(x) u^\sigma = 0 &\text{in } \Omega \\
          u = 0 &\text{on } \partial \Omega
        \end{cases} \]
        by considering
        \begin{equation}
          \begin{cases}
            \Delta u + f(u) = 0 &\text{in } \Omega \\
            u=0 &\text{on } \partial \Omega
          \end{cases}
          \label{eqn:prob2}
        \end{equation}
        where $f(u)$ is continuous in $u \in \R$.

        \begin{enumerate}
          \item
            If $F(u) = \int_{0}^{u} f(t)dt$, use integration by parts to show
            \[n \int_{\Omega}^{} F(u) dx + \int_{\Omega}^{} f(u) \sum_{i=1}^n x_i \frac{\partial u}{\partial x_i} dx = 0 ,\]
            for any $u \in C^1( \overline{\Omega} )$ with $u=0$ on $\partial \Omega$.

          \item
            If $u \in C^2(\Omega) \cap C( \overline{\Omega} )$ satisfies \eqref{eqn:prob2}, then prove Pohozaev's identity
            \[ \frac{n-2}{2} \int_{\Omega}^{} |\nabla u|^2 dx - n \int_{\Omega}^{} F(u)dx + \frac{1}{2} \int_{\partial \Omega}^{} \left(
            \frac{\partial u}{\partial \nu} \right)^2 (x \cdot \nu) ds = 0 \]
            where $\nu$ is the exterior unit normal.

          \item
            When $\Omega$ is a ball in $\R^n$, show that \eqref{eqn:prob2} admits no positive solutions $u \in C^2 ( \Omega ) \cap C(
            \overline{\Omega} )$ when $f(u) = u^\sigma$ with $\sigma \geq (n+2)/(n-2)$.

        \end{enumerate}

      \end{problem}

      \begin{solution}

        \begin{enumerate}
          \item
            By writing $n F(u) = \sum_{i=1}^n F(u) \partial_i x_i$, we get
            \begin{align*}
              n \int_{\Omega}^{} F(u) dx &= \int_{\Omega}^{} \sum_{i=1}^n F(u) \partial_i x_i dx \\
              &= \int_{\partial \Omega}^{} \sum_{i=1}^n F(u) x_i \nu_i ds - \int_{\Omega}^{} \sum_{i=1}^n \partial_i F(u) x_i dx \quad
              \parbox{5cm}{integrating by parts} \\
              &= - \int_{\Omega}^{} \sum_{i=1}^n \partial_i \int_{0}^{u} f(t) dt x_i dx \quad \parbox{5cm}{because $u=0$ on $\partial \Omega$}\\
              &= - \int_{\Omega}^{} \sum_{i=1}^n f(u) \frac{\partial u}{\partial x_i} x_i dx
            \end{align*}

            Therefore,
            \[ n \int_{\Omega}^{} F(u) dx + \int_{\Omega}^{} f(u) \sum_{i=1}^n x_i \frac{\partial u}{\partial x_i} dx = 0 .\]

          \item
            Using part (a) and Einstein summation notation, we have
            \begin{align*}
              n \int_{\Omega}^{} F(u) dx &= - \int_{\Omega}^{} f(u) \partial_i u x_i dx \\
              &= \int_{\Omega}^{} \Delta u \partial_i u x_i dx \quad \parbox{5cm}{by \eqref{eqn:prob2}} \\
              &= \int_{\Omega}^{} \partial_j \partial_j u \partial_i u x_i dx \\
              &= \int_{\partial \Omega}^{} x_i \partial_i u \partial_j u \nu_j ds - \int_{\Omega}^{} \partial_j u \partial_j (x_i \partial_i u) dx \\
              &= \int_{\partial \Omega}^{} x_i \partial_i u \partial_j u \nu_j ds - \int_{\Omega}^{} \partial_j u (\delta_{ij} \partial_i u + x_i
              \partial_j \partial_i u ) dx \\
              &= \int_{\partial \Omega}^{} x_i \partial_i u \partial_j u \nu_j ds - \int_{\Omega}^{} \partial_j u \partial_j u dx -
              \int_{\Omega}^{} x_i \partial_j u \partial_j \partial_i u dx \\
              &= \int_{\partial \Omega}^{} x_i \partial_i u \partial_j u \nu_j ds - \int_{\Omega}^{} | \nabla u |^2 dx - \frac{1}{2}
              \int_{\Omega}^{} x_i \partial_i (\partial_j u \partial_j u) dx \\
              &= \int_{\partial \Omega}^{} x_i \partial_i u \partial_j u \nu_j ds - \int_{\Omega}^{} | \nabla u |^2 dx - - \frac{1}{2} \int_{\partial
              \Omega}^{} x_i \nu_i | \nabla u |^2 ds + \frac{n}{2} \int_{\Omega}^{} | \nabla u |^2 dx \\
              &= \frac{n-2}{2} \int_{\Omega}^{} | \nabla u |^2 dx + \int_{\partial \Omega}^{} x_i \partial_i u \partial_j u \nu_j ds -
              \frac{1}{2} \int_{\partial \Omega}^{} x_i \nu_i | \nabla u |^2 ds \\
              &= \frac{n-2}{2} \int_{\Omega}^{} | \nabla u |^2 dx + \int_{\partial \Omega}^{} (x \cdot \nabla u) (\nabla u \cdot \nu) ds -
              \frac{1}{2} \int_{\partial \Omega}^{} | \nabla u |^2 (x \cdot \nu) ds
            \end{align*}

            Because $u=0$ on $\partial \Omega$, $\partial \Omega$ is a level set for $u$ and $\nu$ is parallel to $\nabla u$. This tells us $\nabla u
            = |\nabla u| \nu$ and $| \nabla u |^2 = \left( \frac{\partial u}{\partial \nu}
            \right)^2$.

            Therefore,
            \begin{equation*}
              \frac{n-2}{2} \int_{\Omega}^{} | \nabla u |^2 dx - n \int_{\Omega}^{} F(u) dx + \frac{1}{2} \int_{\partial \Omega}^{} \left(
              \frac{\partial u}{\partial \nu} \right)^2 ( x \cdot \nu ) ds = 0
            \end{equation*}

          \item
            Multiplying our differential equation by $u$ and integrating, we get
            \begin{align*}
              0 &= \int_{\Omega}^{} u \Delta u + u^{\sigma+1} dx \\
              &= - \int_{\Omega}^{} | \nabla u |^2 dx + \int_{\Omega}^{} u^{\sigma+1} dx
            \end{align*}

            Substituting this into our Pohozaev identity gives
            \begin{align*}
              0 &= \frac{n-2}{2} \int_{\Omega}^{} | \nabla u |^2 dx - n \int_{\Omega}^{} F(u) dx + \frac{1}{2} \int_{\partial \Omega}^{} \left(
              \frac{\partial u}{\partial \nu} \right)^2 (x \cdot \nu) ds \\
              &= \frac{n-2}{2} \int_{\Omega}^{} u^{\sigma + 1} dx - n \int_{\Omega}^{} \int_{0}^{u} t^\sigma dt dx + \frac{1}{2}
              \int_{\partial \Omega}^{} \left( \frac{\partial u}{\partial \nu} \right)^2 (x \cdot \nu) ds \\
              &= \frac{n-2}{2} \int_{\Omega}^{} u^{\sigma+1} dx - \frac{n}{\sigma+1} \int_{\Omega}^{} u^{\sigma + 1} dx + \frac{1}{2}
              \int_{\partial \Omega}^{} \left( \frac{\partial u}{\partial \nu} \right)^2 (x \cdot \nu) ds \\
              &= \left( \frac{n-2}{2} - \frac{n}{\sigma+1} \right) \int_{\Omega}^{} u^{\sigma+1} dx + \frac{1}{2} \int_{\partial \Omega}^{} \left(
              \frac{\partial u}{\partial \nu} \right)^2 (x \cdot \nu) ds
            \end{align*}

            We know $\Omega$ is a ball. Without loss of generality, we assume $\Omega$ is centered at the origin. Then $\nu$ points radially outward
            and is parallel to $x$. Therefore, $(x \cdot \nu) \geq 0$ for all $x \in \partial \Omega$. So
            \begin{equation*}
              0 \geq \left( \frac{n-2}{2} - \frac{n}{\sigma+1} \right) \int_{\Omega}^{} u^{\sigma+1} dx
            \end{equation*}

            If $\sigma > \frac{n+2}{n-2}$, then
            \begin{align*}
              \frac{n-2}{2} - \frac{n}{\sigma + 1} &> \frac{n-2}{2} - \frac{n}{ \frac{n+2}{n-2} + 1} \\
              &= \frac{n-2}{2} - \frac{n}{ \frac{2n}{n-2}} \\
              &= \frac{n-2}{2} - \frac{n-2}{2} \\
              &= 0
            \end{align*}

            Because $u$ is a positive solution, we would have
            \begin{equation*}
              0 < \left( \frac{n-2}{2} - \frac{n}{\sigma + 1} \right) \int_{\Omega}^{} u^{\sigma+1} dx
            \end{equation*}
            which is a contradiction. Therefore, there cannot be any positive solutions with $\sigma < \frac{n+2}{n-2}$.

        \end{enumerate}

      \end{solution}

      \pagebreak

  \item McOwen, Page 237, Problem 7
    \begin{problem}
      Prove the following relationship between the Dirichlet eigenvalues $\lambda_n$ and Neumann eigenvalues $\mu_n$ of the Laplacian on a bounded
      domain with $C^1$-boundary: $\lambda_n \geq \mu_n$ for all $n \geq 1$.
    \end{problem}

    \begin{solution}
      First, we claim the Neumann eigenvalues satisfy a characterization similar to the maximin characterization for Dirichlet eigenvalues.

      Let $v_1, \dots, v_{n-1} \in H^1(\Omega)$. Let $u_i^\ast$ denote the eigenfunctions for the Laplacian with Neumann boundary conditions. Define
      \[ l^\ast \{ v_1, \dots, v_{n-1} \} = \sup_{u \in H^1(\Omega)} \{ \int_{\Omega}^{} | \nabla u |^2 dx : \|u\|_{L^2}=1, \la u, v_i \ra = 0,
      i=1,\dots,n-1 \} \]

      We claim
      \[ \mu_n = \inf_{v_1, \dots, v_{n-1}} l^\ast \{ v_1, \dots, v_{n-1} \} .\]

      We have from problem 5 that
      \[ l \{ u_1^\ast, \dots, u_{n-1}^\ast \} = \mu_n ,\]
      so
      \[ \mu_n \geq \inf_{v_1, \dots, v_{n-1}} l^\ast \{ v_1, \dots, v_{n-1} \} .\]

      For the other inequality, take $v_1, \dots, v_{n-1} \in H^1(\Omega)$ to be arbitrary. Take $u = \sum_{i \geq n} c_i u^\ast_i \in H^1(\Omega)$
      satisfying $\la u, v_i \ra =0$ for $i=1,\dots,n-1$ and $\sum_{i} c_i^2 = 1$. Then by the orthonormality of the $u_i^\ast$ we have
      \begin{align*}
        \int_{\Omega}^{} | \nabla u |^2 dx &= \int_{\Omega}^{} | \sum_{i \geq n} c_i \nabla u_i |^2 dx \\
        &= \sum_{i\geq n} c_i^2 \int_{\Omega}^{} | \nabla u_i |^2 dx \\
        &= \sum_{i \geq n} c_i^2 \mu_i
      \end{align*}

      Combining this with $\mu_1 \leq \mu_2 \leq \dots$, we get
      \begin{align*}
        l^\ast \{ v_1, \dots, v_{n-1} \} &\geq \int_{\Omega}^{} | \nabla u |^2 dx \\
        &= \sum_{i \geq n} \mu_i c_i^2 \\
        &\geq \mu_n \sum_{i \geq n} c_i^2 \\
        &= \mu_n
      \end{align*}

      Therefore,
      \[ \mu_n = \inf_{v_1, \dots, v_{n-1}} l^\ast \{v_1, \dots, v_{n-1}\} \]
      as we claimed.

      By letting $v_1, \dots, v_{n-1} \in H^1_0(\Omega)$ and defining,
      \[ l \{v_1, \dots, v_{n-1} \} = \sup_{u \in H^1_0(\Omega)} \{ \int_{\Omega}^{} | \nabla u |^2 dx : \|u\|_{L^2} = 1, \la u, v_i \ra = 0, i=1,
      \dots, n-1 \} \]
      we get
      \[ \lambda_n = \inf_{v_1, \dots, v_{n-1} \in H^1_0(\Omega)} l \{v_1, \dots, v_{n-1} \} \]
      by the same argument.

      To conclude $\lambda_n \geq \mu_n$ it suffices to take $v_1, \dots, v_{n-1} \in H^1_0(\Omega)$ and find $v_1^\ast, \dots, v_{n-1}^\ast \in
      H^1(\Omega)$ such that $l^\ast \{ v_1^\ast, \dots , v_{n-1}^\ast \} \leq l \{v_1, \dots, v_{n-1}\}$.
      Because $H^1_0(\Omega) \subset H^1(\Omega)$, we can take $v_i^\ast = v_i$ to get the result.

    \end{solution}

    \pagebreak

  \item McOwen Page 227, Problem 8
    \begin{problem}
      Define the graph area functional by
      \[ F(u) = \int_{\Omega}^{} \sqrt{1 + u_x^2 + u_y^2} dx dy \quad \text{for } u \in H^1(\Omega) \]
      where $\Omega$ is a bounded domain in $\R^2$.

      \begin{enumerate}
        \item Compute the derivative of $F$ and show that $F$ is $C^1$ on $H^1(\Omega)$.

        \item
          Given $g \in H^1(\Omega)$, let
          \[ \mathcal{A} = \{ u \in H^1(\Omega) : u-g \in H^1_0(\Omega) \} = \{u = g+v : v \in H^1_0(\Omega) \} .\]
          Show that a critical point of $F$ on $\mathcal{A}$ is a weak solution of the minimal surface equation $(1+u_y^2)u_{xx} - 2 u_x u_y u_{xy} +
          (1+u_x^2)u_{yy} = 0$ in $\Omega, u=g$ on $\partial \Omega$.

        \item
          Generalize (a) and (b) to dimension $n$, obtaining the minimal surface equation given by
          \[ \div \left( \frac{\nabla u}{(1+| \nabla u |^2)^{1/2}} \right) = 0 .\]

      \end{enumerate}

    \end{problem}

    \begin{solution}
      \begin{enumerate}
        \item
          Let $u,v \in H^1(\Omega)$. We first compute
          \begin{align*}
            F'(u)v &= \lim_{\varepsilon \to 0} \frac{1}{\varepsilon} \int_{\Omega}^{} \left( \sqrt{1 + (u_x + \varepsilon v_x)^2 + (u_y + \varepsilon v_y)^2}
            - \sqrt{1 + u_x^2 + u_y^2} \right) dx dy \\
            &= \lim_{\varepsilon \to 0} \frac{1}{\varepsilon} \int_{\Omega}^{} \frac{1 + (u_x + \varepsilon v_x)^2 + (u_y + \varepsilon v_y)^2 - 1 - u_x^2 -
            u_y^2}{\sqrt{1 + (u_x + \varepsilon v_x)^2 + (u_y + \varepsilon v_y)^2} + \sqrt{1 + u_x^2 + u_y^2}} dx dy \\
            &= \lim_{\varepsilon \to 0} \frac{1}{\varepsilon} \int_{\Omega}^{} \frac{2 \varepsilon u_x v_x + 2 \varepsilon u_y v_y + \varepsilon^2 v_x^2 + \varepsilon^2 v_y^2}
            {\sqrt{1 + (u_x + \varepsilon v_x)^2 + (u_y + \varepsilon v_y)^2} + \sqrt{1 + u_x^2 + u_y^2}} dx dy \\
            &= \int_{\Omega}^{} \frac{u_x v_x + u_y v_y}{\sqrt{1 + u_x^2 + u_y^2}} dx dy
          \end{align*}

          Now we must show $F$ is differentiable. Let $u,v \in H^1(\Omega)$. Then
          \begin{align*}
            | F(u+v) &- F(u) - F'(u)v | \\
            &= \left| \int_{\Omega}^{} \sqrt{ 1 + (u_x + v_x)^2 + (u_y + v_y)^2} - \sqrt{1 + u_x^2 + u_x^2}
            - \frac{u_x v_x + u_y v_y}{\sqrt{1 + u_x^2 + u_y^2}} dx dy \right| \\
            &= \left| \int_{\Omega}^{} \frac{\left( (1 + (u_x + v_x)^2 + (u_y + v_y)^2(1+u_x^2+u_y^2) \right)^{1/2} - (1 + u_x^2 + u_y^2) -u_x v_x -
            u_y v_y}{(1+u_x^2+u_y^2)^{1/2}} dx dy \right| \\
            &= \left| \int_{\Omega}^{} \frac{\left( (1+ | \nabla u + \nabla v |^2) (1 + | \nabla u |^2 \right)^{1/2} - (1 + |\nabla u |^2) - \nabla u
            \cdot \nabla v}{( 1 + |\nabla u|^2)^{1/2}} dx \right| \\
            &\leq \left| \int_{\Omega}{} \frac{(1+ | \nabla u | + | \nabla v |)(1 + | \nabla u) - (1 + | \nabla u |^2) - \nabla u \cdot \nabla v}{(1+
            | \nabla u |^2 )^{1/2}} dx \right| \\
            &= \left| \int_{\Omega}^{} \frac{2 | \nabla u | + | \nabla v | + | \nabla u | | \nabla v | - \nabla u \cdot \nabla v}{(1 + | \nabla u
            |^2)^{1/2}} dx \right|
          \end{align*}
          From here, I want this term to be $o( \|v\|_{H^1_0} )$. I don't see how to get to that point, and the fact that I have extra $| \nabla u |$
          terms left makes me think I did something incorrectly.

          To show $F'$ is continuous, take $u,v,w \in H^1_0(\Omega)$. Then
          \begin{align*}
            | (F'(u)-F'(v))w| &= \left| \int_{\Omega}^{} \frac{u_x w_x + u_y w_y}{\sqrt{1 + u_x^2 + u_y^2}} - \frac{v_x w_x + v_y w_y}{\sqrt{1 + v_x^2 +
            v_y^2}} dx dy \right| \\
            &= \left| \int_{\Omega}^{} \frac{\nabla u \cdot \nabla w \sqrt{1+| \nabla v |^2} - \nabla v \cdot \nabla w \sqrt{1 + | \nabla u
            |^2}}{\sqrt{(1 + | \nabla u|^2) (1 + | \nabla v |^2)}} dx dy \right| \\
            &\leq \left| \int_{\Omega}^{} \nabla w \cdot \frac{( \nabla u \sqrt{1 + | \nabla v |^2} - \nabla v \sqrt{1 + | \nabla u |^2} )}{(1+|
              \nabla u |^2)^{1/2} (1 + | \nabla v |^2)^{1/2}} dx dy \right| \\
              &\leq \| w \|_{H^1_0} \left\| \frac{ \nabla u \sqrt{1 + | \nabla v |^2} - \nabla v \sqrt{1 - | \nabla v |^2 }}{(1+ | \nabla u |^2)^{1/2}
              (1 + | \nabla v |^2)^{1/2}} \right\|_{L^2}
          \end{align*}

          I want to say the second term is less than or equal to $\|u-v\|_{H^1_0}$, but I don't see how to do that. Once I have that, I would be able
          to say
          \begin{align*}
            \| F'(u) - F'(v) \|_{H^{-1}} &= \sup_{w \not= 0} \frac{| (F'(u) - F'(v))w|}{\|w\|_{H^1_0}} \\
            &\leq \|u-v\|_{H^1_0}
          \end{align*}
          and $F'$ is therefore continuous.

        \item
          We see that $(1 + u_y^2) u_{xx} - 2 u_x u_x u_{xy} + (1+u_x^2) u_{yy} = 0$ if and only if
        \begin{align*}
          0 &= \frac{u_{xx} (1+u_y^2) - 2 u_x u_y u_{xy} + u_{yy}(1 + u_x^2)}{(1 + u_x^2 + u_y^2)^{3/2}} \\
          &= \frac{(u_{xx} + u_{yy}) (1 + u_x^2 + u_y^2)^{1/2} - (u_x^2 u_{xx} + 2 u_x u_y u_{xy} + u_y u_{yy}) (1 + u_x^2 + u_y^2)^{-1/2}}{1 + u_x^2
          + u_y^2} \\
          &= \frac{u_xx ( 1 + u_x^2 + u_y^2 )^{1/2} - u_x ( 1 + u_x^2 + u_y^2 )^{-1/2} (u_x u_{xx} + u_y u_{xy})}{1+u_x^2+u_y^2} \\
          &\quad + \frac{u_{yy} ( 1 + u_x^2 + u_y^2)^{1/2} - u_y (1+u_x^2+u_y^2)^{-1/2} (u_x u_{xy} + u_y u_{yy})}{1 + u_x^2 + u_y^2} \\
          &= \partial_x \left( \frac{u_x}{(1 + u_x^2 + u_y^2)^{1/2}} \right) + \partial_y \left( \frac{u_y}{(1 + u_x^2 + u_y^2)^{1/2}} \right) \\
          &= \div \left( \frac{\nabla u}{(1 + | \nabla u|^2)^{1/2}} \right)
        \end{align*}

        Multiplying by $v \in H^1_0(\Omega)$ and integrating, we get
        \begin{align*}
          0 &= \int_{\Omega}^{} \div \left( \frac{\nabla u}{( 1 + | \nabla u |^2 )^{1/2}} \right) v dx \\
          &= - \int_{\Omega}^{} \frac{\nabla u}{( 1+ | \nabla u |^2 )^{1/2}} \cdot \nabla v dx \\
          &= - \int_{\Omega}^{} \frac{u_x v_x + u_y v_y}{\sqrt{1 + u_x^2 + u_y^2}} dx
        \end{align*}

        Multiplying by -1, we see critical points of $F$ are weak solutions of the minimal surface equation.

        \item
          In the general case, we have the functional
          \[ J(u) = \int_{\Omega}^{} \sqrt{ 1 + | \nabla u |^2 } dx .\]
          Take $u,v \in H^1(\Omega)$. Then
          \begin{align*}
            J'(u)v &= \lim_{\varepsilon \to 0} \frac{1}{\varepsilon} \int_{\Omega}^{} \sqrt{ 1 + | \nabla u + \varepsilon \nabla v |^2 } - \sqrt{ 1 + | \nabla u
            |^2} dx \\
            &= \lim_{\varepsilon \to 0} \frac{1}{\varepsilon} \int_{\Omega}^{} \frac{1 + | \nabla u + \varepsilon \nabla v |^2 - 1 - |\nabla
            u|^2}{\sqrt{1+ | \nabla u + \varepsilon \nabla v |^2} + \sqrt{ 1 + | \nabla u |^2}} dx \\
            &= \lim_{\varepsilon \to 0} \frac{1}{\varepsilon} \int_{\Omega}^{} \frac{2 \varepsilon \nabla u \cdot \nabla v + \varepsilon^2 | \nabla v
            |^2}{\sqrt{1 + | \nabla u + \varepsilon \nabla v |^2} + \sqrt{1 + | \nabla u |^2}} dx \\
            &= \int_{\Omega}^{} \frac{\nabla u \cdot \nabla v}{\sqrt{1 + | \nabla u |^2}} dx
          \end{align*}

          The calculations to show $J$ is $C^1$ are exactly the same as in part (a).

          Let $v \in H^1_0(\Omega)$. Then
          \begin{align*}
            0 &= \int_{\Omega}^{} \div \left( \frac{\nabla u}{( 1 + | \nabla u |^2 )^{1/2}} \right) v dx \\
            &= - \int_{\Omega}^{} \frac{\nabla u \cdot \nabla v}{ (1 + | \nabla u |^2 )^{1/2}} dx
          \end{align*}

          Therefore, we see critical points of $J$ are weak solutions of the minimal surface equation.

      \end{enumerate}
    \end{solution}

    \pagebreak

  \item McOwen Page 228, Problem 11
    \begin{problem}
      Suppose $F$ is a $C^1$ functional on $X$ satisfying (i) $F'(u) = L + K(u)$, where $L:X \to X^\ast$ is an isomorphism (i.e., a bounded linear
      operator that is one-to-one and onto) and $K:X \to X^\ast$ is a compact map (i.e., maps bounded sets of $X$ to precompact sets of $X^\ast$ but
      $K$ is not necessarily linear); and (ii) every Palais-Smale sequence is bounded in $X$. Show that $F$ satisfies the Palais-Smale condition.
    \end{problem}

    \begin{solution}
      Let $\{x_m\}$ be a Palais-Smale sequence. Then by assumption, $\{x_m\}$ is bounded in $X$. Because $K$ is compact, $K(u_m)$ contains a
      convergent subsequence in $X^\ast$. We will denote this subsequence $K(u_{m_k})$ and let $K(u^\ast)$ be its limit.

      By assumption, we know
      \[ F'(u_{m_k}) = K(u_{m_k}) + L u_{m_k} \to 0 \text{ in } X^\ast .\]
      Applying $L^{-1}$, we get the sequence $u_{m_k} + L^{-1} K (u_{m_k})$ in $X$ that satisfies
      \[ L( u_{m_k} + L^{-1} K(u_{m_k})) \to 0 \text{ in } X^\ast .\]
      Then we have
      \begin{align*}
        \| u_{m_k} + L^{-1} K (u_{m_k}) \|_X &= \| L^{-1} L ( u_{m_k} + L^{-1} K (u_{m_k})) \|_X \\
        &\leq C \| L( u_{m_k} + L^{-1} K ( u_{m_k})) \|_{X^\ast} \quad \parbox{5cm}{by boundedness of $L^{-1}$} \\
        &\quad \to 0
      \end{align*}

      Therefore, $u_{m_k} + L^{-1} K (u_{m_k}) \to 0$ in $X$. Looking at $L^{-1} K(u_{m_k})$, we have
      \begin{align*}
        \| L^{-1} K(u_{m_k}) - L^{-1} K(u^\ast) \|_X &= \| L^{-1} ( K(u_{m_k}) - K(u^\ast) ) \|_X \\
        &\leq C \| K(u_{m_k}) - K(u^\ast) \|_{X^\ast} \quad \parbox{5cm}{by boundedness of $L^{-1}$} \\
        &\quad \to 0
      \end{align*}

      Thus $L^{-1} K(u_{m_k})$ converges in $X$. By the convergence of $u_{m_k} + L^{-1} K( u_{m_k})$ in $X$, we now know $u_{m_k}$ converges in $X$.
      This shows $\{x_m\}$ contains a convergent subsequence, and $F$ satisfies the Palais-Smale condition.
    \end{solution}

\end{enumerate}
\end{document}


